% Options for packages loaded elsewhere
% Options for packages loaded elsewhere
\PassOptionsToPackage{unicode}{hyperref}
\PassOptionsToPackage{hyphens}{url}
%
\documentclass[
  ignorenonframetext,
  aspectratio=169,
]{beamer}
\newif\ifbibliography
\usepackage{pgfpages}
\setbeamertemplate{caption}[numbered]
\setbeamertemplate{caption label separator}{: }
\setbeamercolor{caption name}{fg=normal text.fg}
\beamertemplatenavigationsymbolsempty
% remove section numbering
\setbeamertemplate{part page}{
  \centering
  \begin{beamercolorbox}[sep=16pt,center]{part title}
    \usebeamerfont{part title}\insertpart\par
  \end{beamercolorbox}
}
\setbeamertemplate{section page}{
  \centering
  \begin{beamercolorbox}[sep=12pt,center]{section title}
    \usebeamerfont{section title}\insertsection\par
  \end{beamercolorbox}
}
\setbeamertemplate{subsection page}{
  \centering
  \begin{beamercolorbox}[sep=8pt,center]{subsection title}
    \usebeamerfont{subsection title}\insertsubsection\par
  \end{beamercolorbox}
}
% Prevent slide breaks in the middle of a paragraph
\widowpenalties 1 10000
\raggedbottom
\AtBeginPart{
  \frame{\partpage}
}
\AtBeginSection{
  \ifbibliography
  \else
    \frame{\sectionpage}
  \fi
}
\AtBeginSubsection{
  \frame{\subsectionpage}
}
\usepackage{iftex}
\ifPDFTeX
  \usepackage[T1]{fontenc}
  \usepackage[utf8]{inputenc}
  \usepackage{textcomp} % provide euro and other symbols
\else % if luatex or xetex
  \usepackage{unicode-math} % this also loads fontspec
  \defaultfontfeatures{Scale=MatchLowercase}
  \defaultfontfeatures[\rmfamily]{Ligatures=TeX,Scale=1}
\fi
\usepackage{lmodern}

\ifPDFTeX\else
  % xetex/luatex font selection
\fi
% Use upquote if available, for straight quotes in verbatim environments
\IfFileExists{upquote.sty}{\usepackage{upquote}}{}
\IfFileExists{microtype.sty}{% use microtype if available
  \usepackage[]{microtype}
  \UseMicrotypeSet[protrusion]{basicmath} % disable protrusion for tt fonts
}{}
\makeatletter
\@ifundefined{KOMAClassName}{% if non-KOMA class
  \IfFileExists{parskip.sty}{%
    \usepackage{parskip}
  }{% else
    \setlength{\parindent}{0pt}
    \setlength{\parskip}{6pt plus 2pt minus 1pt}}
}{% if KOMA class
  \KOMAoptions{parskip=half}}
\makeatother


\usepackage{longtable,booktabs,array}
\usepackage{calc} % for calculating minipage widths
\usepackage{caption}
% Make caption package work with longtable
\makeatletter
\def\fnum@table{\tablename~\thetable}
\makeatother
\usepackage{graphicx}
\makeatletter
\newsavebox\pandoc@box
\newcommand*\pandocbounded[1]{% scales image to fit in text height/width
  \sbox\pandoc@box{#1}%
  \Gscale@div\@tempa{\textheight}{\dimexpr\ht\pandoc@box+\dp\pandoc@box\relax}%
  \Gscale@div\@tempb{\linewidth}{\wd\pandoc@box}%
  \ifdim\@tempb\p@<\@tempa\p@\let\@tempa\@tempb\fi% select the smaller of both
  \ifdim\@tempa\p@<\p@\scalebox{\@tempa}{\usebox\pandoc@box}%
  \else\usebox{\pandoc@box}%
  \fi%
}
% Set default figure placement to htbp
\def\fps@figure{htbp}
\makeatother





\setlength{\emergencystretch}{3em} % prevent overfull lines

\providecommand{\tightlist}{%
  \setlength{\itemsep}{0pt}\setlength{\parskip}{0pt}}



 


\makeatletter
\@ifpackageloaded{caption}{}{\usepackage{caption}}
\AtBeginDocument{%
\ifdefined\contentsname
  \renewcommand*\contentsname{Table of contents}
\else
  \newcommand\contentsname{Table of contents}
\fi
\ifdefined\listfigurename
  \renewcommand*\listfigurename{List of Figures}
\else
  \newcommand\listfigurename{List of Figures}
\fi
\ifdefined\listtablename
  \renewcommand*\listtablename{List of Tables}
\else
  \newcommand\listtablename{List of Tables}
\fi
\ifdefined\figurename
  \renewcommand*\figurename{Figure}
\else
  \newcommand\figurename{Figure}
\fi
\ifdefined\tablename
  \renewcommand*\tablename{Table}
\else
  \newcommand\tablename{Table}
\fi
}
\@ifpackageloaded{float}{}{\usepackage{float}}
\floatstyle{ruled}
\@ifundefined{c@chapter}{\newfloat{codelisting}{h}{lop}}{\newfloat{codelisting}{h}{lop}[chapter]}
\floatname{codelisting}{Listing}
\newcommand*\listoflistings{\listof{codelisting}{List of Listings}}
\makeatother
\makeatletter
\makeatother
\makeatletter
\@ifpackageloaded{caption}{}{\usepackage{caption}}
\@ifpackageloaded{subcaption}{}{\usepackage{subcaption}}
\makeatother

\usepackage{bookmark}
\IfFileExists{xurl.sty}{\usepackage{xurl}}{} % add URL line breaks if available
\urlstyle{same}
\hypersetup{
  pdftitle={Transformação de Variáveis Unidimensionais},
  hidelinks,
  pdfcreator={LaTeX via pandoc}}



\title{Transformação de Variáveis Unidimensionais}
\author{}
\date{}

\begin{document}
\frame{\titlepage}


\begin{frame}{Função de Variável Aleatória}
\phantomsection\label{funuxe7uxe3o-de-variuxe1vel-aleatuxf3ria}
Se \(X\) é uma variável aleatória com função densidade acumulada
\(F_x(x)\), então qualquer função de \(X\), por exemplo, \(g(X)\) também
é uma variável aleatória.

\pause

Geralmente, \(g(X)\) é de nosso interesse e denotamos \(Y = g(X)\) para
indicar a nova variável aleatória \(g(X)\).

\pause

Uma vez que \(Y\) é uma função de \(X\), podemos descrever o
comportamento probabilístico de \(Y\) em termos de \(X\), isto é, para
um dado conjunto \(A\),

\[P(Y \in A) = P(g(X) \in A)\]

mostrando que a distribuição de \(Y\) depende das funções \(F_X\) e
\(g\).
\end{frame}

\begin{frame}{Função de Variável Aleatória}
\phantomsection\label{funuxe7uxe3o-de-variuxe1vel-aleatuxf3ria-1}
Formalmente, ao escrevermos \(y = g(x)\), a função \(g(x)\) define uma
função no suporte original de \(X\), denotado por \(S_X\), para um novo
espaço amostral, denotado por \(S_Y\), o suporte de \(Y\).

\pause

Seja então a função inversa de \(g\), denotada por \(g^{-1}\), que é uma
função de \(S_y\) para \(S_x\).

\[g^{-1}(y)=\{x : g(x) = y \}\]

\pause

Portanto, a função inversa consiste no conjunto de valores de \(X\),
para os quais \(g(x) = y\), dado um valor de \(y\) fixado do suporte de
\(Y\), que é a variável aleatória obtida pela transformação de
interesse, \(g(X)\).
\end{frame}

\begin{frame}{Função de Variável Aleatória}
\phantomsection\label{funuxe7uxe3o-de-variuxe1vel-aleatuxf3ria-2}
É importante observar que cada valor de \(x\) pertencente ao suporte de
\(X\), corresponde a um único valor de \(y\) pertencente ao suporte de
\(Y\). Por outro lado, um valor \(y \in S_Y\) pode corresponder a mais
de um valor do suporte de \(X\), conforme ilustra a figura abaixo

\begin{center}
\pandocbounded{\includegraphics[keepaspectratio]{../../images/funcao_var_alea.png}}
\end{center}
\end{frame}

\begin{frame}{Função de Variável Aleatória}
\phantomsection\label{funuxe7uxe3o-de-variuxe1vel-aleatuxf3ria-3}
Finalmente, se a variável aleatória \(Y\) for definida por \(Y= g(X)\),
podemos escrever para qualquer conjunto \(A \subset S_Y\),

\[
\begin{aligned}
P(Y \in A) &= P(g(X) \in A) \\
           &= P(\{x \in S_X : g(x) \in A\}) \\
           &= P(X \in g^{-1}(A))
\end{aligned}
\]

Isto define a distribuição de probabilidade de \(Y\).
\end{frame}

\begin{frame}{Função de Variável Aleatória: Caso Discreto}
\phantomsection\label{funuxe7uxe3o-de-variuxe1vel-aleatuxf3ria-caso-discreto}
Se \(X\) for uma variável aleatória discreta e \(Y= g(X)\), então \(Y\)
também será uma variável aleatória discreta. Assim,

\[p_Y(y) = P(Y = y) = \sum_{x \in g^{-1}(y)} p_X(x), \qquad \text{para} \,\, y \in S_Y\]
e \(p_Y(y) = 0\) para \(y \notin S_Y\).
\end{frame}

\begin{frame}{Função de Variável Aleatória: Caso Discreto}
\phantomsection\label{funuxe7uxe3o-de-variuxe1vel-aleatuxf3ria-caso-discreto-1}
\textbf{Exemplo 01:} Suponhamos que a distribuição de probabilidades da
variável aleatória \(X\) é dada pela tabela abaixo:

\begin{longtable}[]{@{}lllllll@{}}
\toprule\noalign{}
\(x\) & -2 & -1 & 0 & 1 & 2 & 3 \\
\midrule\noalign{}
\endhead
\(p_X(x)\) & 0,1 & 0,2 & 0,2 & 0,3 & 0,1 & 0,1 \\
\bottomrule\noalign{}
\end{longtable}

Seja \(Y = 3X+1\). Encontre a distribuição da variável aleatória \(Y\).

\pause

\textbf{Solução:}

Como \(Y = 3X+1\),

\begin{longtable}[]{@{}lllllll@{}}
\toprule\noalign{}
\(y = 3x + 1\) & -5 & -2 & 1 & 4 & 7 & 10 \\
\midrule\noalign{}
\endhead
\(p_Y(y)\) & 0,1 & 0,2 & 0,2 & 0,3 & 0,1 & 0,1 \\
\bottomrule\noalign{}
\end{longtable}
\end{frame}

\begin{frame}{Função de Variável Aleatória: Caso Discreto}
\phantomsection\label{funuxe7uxe3o-de-variuxe1vel-aleatuxf3ria-caso-discreto-2}
\textbf{Exemplo 02:} Considere a mesma distribuição de probabilidades da
variável aleatória \(X\), dada pela tabela abaixo:

\begin{longtable}[]{@{}lllllll@{}}
\toprule\noalign{}
\(x\) & -2 & -1 & 0 & 1 & 2 & 3 \\
\midrule\noalign{}
\endhead
\(p_X(x)\) & 0,1 & 0,2 & 0,2 & 0,3 & 0,1 & 0,1 \\
\bottomrule\noalign{}
\end{longtable}

Contudo, vamos agora definir a variável aleatória \(Y = X^2\). Qual a
distribuição de probabilidades de \(Y\)?

\pause

\textbf{Solução:}

Temos que \(S_X = \{-2, -1, 0, 1, 2, 3\}\). Como \(Y = X^2\), temos que
o suporte de \(Y\) é dado por \(S_Y = \{0, 1, 4, 9\}\), de forma que a
distribuição da variável \(Y\) será dada por:
\end{frame}

\begin{frame}{Função de Variável Aleatória: Caso Discreto}
\phantomsection\label{funuxe7uxe3o-de-variuxe1vel-aleatuxf3ria-caso-discreto-3}
\[
\begin{aligned}
P(Y = 0) &= P(X = 0) = 0{,}2 \\[6pt]
P(Y = 1) &= P(X = -1) + P(X = 1) = 0{,}5 \\[6pt]
P(Y = 4) &= P(X = -2) + P(X = 2) = 0{,}2 \\[6pt]
P(Y = 9) &= P(X = 3) = 0{,}1
\end{aligned}
\]

Logo,

\begin{longtable}[]{@{}lllll@{}}
\toprule\noalign{}
\(y = x^{2}\) & 0 & 1 & 4 & 9 \\
\midrule\noalign{}
\endhead
\(p_Y(y)\) & 0,2 & 0,5 & 0,2 & 0,1 \\
\bottomrule\noalign{}
\end{longtable}
\end{frame}

\begin{frame}{Função de Variável Aleatória: Caso Discreto}
\phantomsection\label{funuxe7uxe3o-de-variuxe1vel-aleatuxf3ria-caso-discreto-4}
\textbf{Exemplo 03:} Suponha que \(X \sim Poisson(\lambda)\), com função
de probabilidade

\[
P(x | \lambda)=
  \begin{cases}
\dfrac{e^{-\lambda} \lambda^x}{x!}, & x \in \{0, 1, 2, 3, \cdots \} \\[6pt]
0, & c.c.
\end{cases}
\]

Considere que \(Y = g(X)\), tal que \(g(X) = 0\), se \(x\) é par e
\(g(X) = 1\), se \(x\) é ímpar. Obter a função de probabilidade de
\(Y\).

\pause

\textbf{Solução:}

Temos que o conjunto suporte de \(Y\) é \(S_Y = \{0,1\}\). Vamos
calcular a \(P(Y = 0)\).
\end{frame}

\begin{frame}{Função de Variável Aleatória: Caso Discreto}
\phantomsection\label{funuxe7uxe3o-de-variuxe1vel-aleatuxf3ria-caso-discreto-5}
\[
\begin{aligned}
P(Y = 0)
 &= P(g(X) = 0)
  = P(X \text{ ser par}) \\[6pt]
 &= \sum_{x \in \{0,2,4,\ldots\}}
    \frac{e^{-\lambda}\lambda^{x}}{x!} \\[6pt]
 &= e^{-\lambda}
    \sum_{k = 0}^{\infty}
    \frac{\lambda^{2k}}{(2k)!}.
\end{aligned}
\]
\end{frame}

\begin{frame}{Aparte --- Séries de Taylor e Maclaurin}
\phantomsection\label{aparte-suxe9ries-de-taylor-e-maclaurin}
\begin{block}{Série de Taylor}
\phantomsection\label{suxe9rie-de-taylor}
Expansão de uma função em torno de um ponto \(a\):

A expansão de Taylor escrita termo a termo é:

\[
{\small
\begin{aligned}
    f(x) &= f(a)
    + f'(a)(x-a)
+ \frac{f''(a)}{2!}(x-a)^2
+ \frac{f^{(3)}(a)}{3!}(x-a)^3 + \cdots = \sum_{n=0}^{\infty} \frac{f^{(n)}(a)}{n!}(x-a)^n
\end{aligned}
}
\]
\end{block}
\end{frame}

\begin{frame}{Aparte --- Séries de Taylor e Maclaurin}
\phantomsection\label{aparte-suxe9ries-de-taylor-e-maclaurin-1}
\begin{block}{Série de Maclaurin}
\phantomsection\label{suxe9rie-de-maclaurin}
Caso particular da série de Taylor com \(a = 0\):

\[ f(x) = \sum_{n=0}^{\infty} \frac{f^{(n)}(0)}{n!}x^n \]

Exemplos:

\[
e^x
= 1
+ x
+ \frac{x^2}{2!}
+ \frac{x^3}{3!}
+ \frac{x^4}{4!}
+ \frac{x^5}{5!}
+ \cdots
+ \frac{x^n}{n!}
+ \cdots= \sum_{n=0}^{\infty} \frac{x^n}{n!}
\]

\[
e^{-x}
= 1
- x
+ \frac{x^2}{2!}
- \frac{x^3}{3!}
+ \frac{x^4}{4!}
- \frac{x^5}{5!}
+ \cdots
+ (-1)^n \frac{x^n}{n!}
+ \cdots = \sum_{n=0}^{\infty} (-1)^n \frac{x^n}{n!}
\]
\end{block}
\end{frame}

\begin{frame}{Função de Variável Aleatória: Caso Discreto}
\phantomsection\label{funuxe7uxe3o-de-variuxe1vel-aleatuxf3ria-caso-discreto-6}
Assim,
\(e^{\lambda} + e^{-\lambda} = 2\sum_{k=0}^{\infty} \frac{\lambda^{2k}}{(2k)!}\)
e,

\[
\begin{aligned}
P(Y = 0)
 &= e^{-\lambda} \sum_{k = 0}^{\infty} \frac{\lambda^{2k}}{(2k)!} \\[6pt]
 &= e^{-\lambda} \cdot \frac{1}{2}\Big(e^{\lambda} + e^{-\lambda}\Big) \\[6pt]
 &= \frac{1}{2}\Big(1 + e^{-2\lambda}\Big).
\end{aligned}
\]
\end{frame}

\begin{frame}{Função de Variável Aleatória: Caso Discreto}
\phantomsection\label{funuxe7uxe3o-de-variuxe1vel-aleatuxf3ria-caso-discreto-7}
Como \(\sum_{y=0}^{\infty} P(Y = y) = 1\), temos que

\[
\begin{aligned}
P(Y = 1)
 &= P(X \text{ ser ímpar})
  = 1 - \frac{1}{2}\Big(1 + e^{-2\lambda}\Big) \\[6pt]
 &= \frac{1}{2}\Big(1 - e^{-2\lambda}\Big).
\end{aligned}
\]

\pause

Finalmente, \(p_Y(y) = 0\) para outros valores de \(y \notin S_Y\).
Portanto, \(Y \sim Bernoulli(p)\) com

\[p = \dfrac{1}{2}\Big(1 - e^{-2\lambda}\Big), \qquad \lambda > 0\]
\end{frame}

\begin{frame}{Função de Variável Aleatória: Caso Discreto}
\phantomsection\label{funuxe7uxe3o-de-variuxe1vel-aleatuxf3ria-caso-discreto-8}
\textbf{Exemplo 04:} Considere que \(X \sim Binomial (n,p)\) com função
de probabilidade

\[
P(x | n,p)=
\begin{cases}
\dbinom{n}{x}\, p^{x}\, (1-p)^{\,n-x}, \qquad x \in \{0, 1, 2, \dots, n\}  \\[6pt]
0, & c.c.
\end{cases}
\]

Considere \(Y = g(X) = n - X\) e obtenha a distribuição de probabilidade
de \(Y\).

\pause

\textbf{Solução:}

Considerando \(Y = g(X) = n - X\), temos que
\(S_Y = \{0, 1, 2, \dots, n\}\). Quando \(Y =  y\), temos \(X = n-y\),
assim,
\end{frame}

\begin{frame}{Função de Variável Aleatória: Caso Discreto}
\phantomsection\label{funuxe7uxe3o-de-variuxe1vel-aleatuxf3ria-caso-discreto-9}
\[
\begin{aligned}
P(Y = y) &= P(g(X) = y) = P(X = g^{-1}(y)) = P(X = n - y) \\
         &= \binom{n}{\,n-y\,}\, p^{\,n-y}\, (1-p)^{\,n-(n-y)} \\
         &= \binom{n}{\,n-y\,}\, p^{\,n-y}\, (1-p)^{\,y} \\
         &= \binom{n}{y}\, (1-p)^{y}\, p^{\,n-y}, 
         \qquad \text{uma vez que } \binom{n}{n-y} = \binom{n}{y}
\end{aligned}
\]

que equivale a dizer que \(Y \sim binomial(n, 1-p)\).
\end{frame}

\begin{frame}{Função de Variável Aleatória: Caso Contínuo}
\phantomsection\label{funuxe7uxe3o-de-variuxe1vel-aleatuxf3ria-caso-contuxednuo}
\begin{block}{Método da Função de Distribuição}
\phantomsection\label{muxe9todo-da-funuxe7uxe3o-de-distribuiuxe7uxe3o}
\textbf{Teorema:} Consideremos uma variável aleatória contínua \(X\) com
função de distribuição de probabilidade \(F_X\) conhecida e a função
\(g: \mathbb{R} \rightarrow \mathbb{R}\), então \(Y = g(X)\) é uma
variável aleatória com função distribuição dada por

\[F_Y(y) = P(Y \le y) = P(X \in g^{-1}\big((-\infty, y]\big) = \int_{g^{-1}\big((-\infty, y]\big)}f_X(x)\,\,dx\]

em que \(g^{-1}(y)\) é a função inversa de \(g\).
\end{block}
\end{frame}

\begin{frame}{Função de Variável Aleatória: Caso Contínuo}
\phantomsection\label{funuxe7uxe3o-de-variuxe1vel-aleatuxf3ria-caso-contuxednuo-1}
\begin{itemize}
\item
  Esse método consiste em encontrar a função de distribuição da variável
  transformada \(Y = g(X)\) a partir da função de distribuição de \(X\).
\item
  Se \(Y\) for variável aleatória continua, podemos então obter a sua
  função densidade a partir da derivada da função de distribuição já
  encontrada.
\item
  Não necessariamente \(Y = g(X)\) será variável aleatória contínua,
  \(Y\) por ser contínua, discreta e até mesmo uma variável aleatória
  que não seja nem discreta nem contínua.
\end{itemize}
\end{frame}

\begin{frame}{Função de Variável Aleatória: Caso Contínuo}
\phantomsection\label{funuxe7uxe3o-de-variuxe1vel-aleatuxf3ria-caso-contuxednuo-2}
\textbf{Exemplo 05:} Considere que \(X \sim Exp(\lambda)\), sendo
\(\lambda > 0\) conhecido, com função densidade de probabilidade dada
por

\[
f_X(x| \lambda) =
\begin{cases}
\lambda \,\, e^{-\lambda x} & \text{para } x > 0  \\[6pt] 
0, & c.c.
\end{cases}
\] Para \(Y = g(X)\) dada por

\[
Y = g(X) =
\begin{cases}
0 & \text{se }  X \le \dfrac{1}{\lambda}  \\[6pt] 
1, & c.c.
\end{cases}
\] obter a distribuição de \(Y\).
\end{frame}

\begin{frame}{Função de Variável Aleatória: Caso Contínuo}
\phantomsection\label{funuxe7uxe3o-de-variuxe1vel-aleatuxf3ria-caso-contuxednuo-3}
\textbf{Solução:}

Como o evento \(\{Y = 0\}\) ocorre quando \(X \le 1/\lambda\) e a função
de distribuição exponencial é \(F_X(x) = 1 - \exp(-\lambda x)\), então

\[P(Y=0) = F_X\Bigg( \dfrac{1}{\lambda}\Bigg) = 1 - \exp \Bigg(-\lambda \times \dfrac{1}{\lambda}\Bigg) = 1-e^{-1}\]

Temos então, que

\[P(Y = 1) = 1 - F_X\Bigg( \dfrac{1}{\lambda}\Bigg) = 1 - (1 - e^{-1}) = e^{-1}\]
\end{frame}

\begin{frame}{Função de Variável Aleatória: Caso Contínuo}
\phantomsection\label{funuxe7uxe3o-de-variuxe1vel-aleatuxf3ria-caso-contuxednuo-4}
Logo, \(Y\) é uma variável aleatória discreta, pois \(S_Y = \{0,1\}\).
Além disso, \(Y \sim Bernoulli(p = e^{-1})\), com \emph{fp} dada por

\[
p_Y(y) =
\begin{cases}
p^y(1-p)^{1-y} & \text{para } y \in \{0,1\}  \\[6pt] 
0, & c.c.
\end{cases}
\] com \(p = e^{-1}\).
\end{frame}

\begin{frame}{Função de Variável Aleatória: Caso Contínuo}
\phantomsection\label{funuxe7uxe3o-de-variuxe1vel-aleatuxf3ria-caso-contuxednuo-5}
\textbf{Exemplo 06:} Seja \(X\) uma variável aleatória tal que
\(X \sim N(0,1)\), cuja \emph{fdp} é dada por

\[f_X(x) = \dfrac{1}{\sqrt{2\pi}}
\exp\left( -\dfrac{x^2}{2} \right), \quad x \in \mathbb{R}\]

Considerando \(Y = g(X) = X^2\), obter a função densidade de
probabilidade de \(Y\).
\end{frame}

\begin{frame}{Função de Variável Aleatória: Caso Contínuo}
\phantomsection\label{funuxe7uxe3o-de-variuxe1vel-aleatuxf3ria-caso-contuxednuo-6}
\textbf{Solução:}

Temos que o evento \(\{Y \le y\}\), para \(y\) fixado e pertencente ao
conjunto suporte \(S_Y = [0, \infty) \subset \mathbb{R}\), cuja
equivalência em relação aos eventos pertencentes ao suporte de \(X\),
\(S_X = \mathbb{R}\), é dada por
\(\{Y \le y\} = \{-\sqrt{y} \le X \le \sqrt{y}\}\), conforme ilustrado
abaixo

\begin{center}
\includegraphics[width=0.7\linewidth,height=\textheight,keepaspectratio]{transformacoes_files/figure-beamer/unnamed-chunk-1-1.pdf}
\end{center}
\end{frame}

\begin{frame}{Função de Variável Aleatória: Caso Contínuo}
\phantomsection\label{funuxe7uxe3o-de-variuxe1vel-aleatuxf3ria-caso-contuxednuo-7}
Portanto,

\[F_Y(y) = P(Y \le y) = P(-\sqrt{y} \le X \le \sqrt{y}) = \Phi_X(\sqrt{y}) - \Phi_X(-\sqrt{y})\]
para \(y > 0\), sendo \(\Phi_X\), a função de distribuição de
probabilidade da normal padrão.
\end{frame}

\begin{frame}{Função de Variável Aleatória: Caso Contínuo}
\phantomsection\label{funuxe7uxe3o-de-variuxe1vel-aleatuxf3ria-caso-contuxednuo-8}
Se derivarmos \(F_y(y)\) em relação a \(y\), e usando o fato de
\(\Gamma\Big(\frac{1}{2}\Big) =  \sqrt{\pi}\), temos

\[\small
\begin{aligned}
f_Y(y)
 &= \frac{dF_Y(y)}{dy}
   = \frac{d\Big[ \Phi_X(\sqrt{y}) - \Phi_X(-\sqrt{y}) \Big]}{dy} \\[6pt]
 &= f_X(\sqrt{y}) \cdot \frac{1}{2\sqrt{y}}
  + f_X(-\sqrt{y}) \cdot \frac{1}{2\sqrt{y}}
   \qquad \text{(regra da cadeia)} \\[6pt]
 &= f_X(\sqrt{y}) \cdot \frac{1}{\sqrt{y}}
   \qquad \text{(simetria da normal)} \\[6pt]
 &= \frac{1}{\sqrt{2\pi}}
    \exp\!\left( -\frac{(\sqrt{y})^{2}}{2}\right)
    \cdot \frac{1}{\sqrt{y}} = \frac{1}{2^{1/2}\sqrt{\pi}}\,e^{-y/2}\,y^{-1/2} = \chi^2_1(y).
\end{aligned}
\]
\end{frame}

\begin{frame}{Função de Variável Aleatória: Caso Contínuo}
\phantomsection\label{funuxe7uxe3o-de-variuxe1vel-aleatuxf3ria-caso-contuxednuo-9}
\begin{block}{Método do Jacobiano}
\phantomsection\label{muxe9todo-do-jacobiano}
\textbf{Teorema:} Seja \(X\) com \emph{fdp} \(f_X(x)\) e \(Y=g(X)\),
onde \(g\) é uma função monótona. Sejam \(S_X = \{x: f_X(x) > 0\}\) e
\(S_Y = \{y: y = g(x) \,\, \text{para algum } \, x \in S_X\}\). Suponha
que \(f_X(x)\) seja contínua em \(S_X\) e que \(g^{-1}(y)\) tenha uma
derivada contínua em \(S_Y\). Então,

\[
f_Y(y) =
\begin{cases}
f_X \Big(g^{-1}(y)\Big) \Bigg|\dfrac{d}{dy}g^{-1}(y)\Bigg| & y \in S_Y  \\[6pt] 
0, & c.c.
\end{cases}
\]
\end{block}
\end{frame}

\begin{frame}{Função de Variável Aleatória: Caso Contínuo}
\phantomsection\label{funuxe7uxe3o-de-variuxe1vel-aleatuxf3ria-caso-contuxednuo-10}
\textbf{Exemplo 07:} Suponha uma variável aleatória contínua \[
X \sim N(\mu, \sigma^2), \quad \sigma > 0.
\]

Sua função densidade é \[
f_X(x) = \frac{1}{\sigma \sqrt{2\pi}}
\exp\left( -\frac{(x - \mu)^2}{2\sigma^2} \right), \quad x \in \mathbb{R}
\]

Definimos a nova variável aleatória \[
Y = \frac{X - \mu}{\sigma}.
\]

Queremos encontrar a distribuição de \(Y\) usando o \textbf{método do
jacobiano}.
\end{frame}

\begin{frame}{Função de Variável Aleatória: Caso Contínuo}
\phantomsection\label{funuxe7uxe3o-de-variuxe1vel-aleatuxf3ria-caso-contuxednuo-11}
\textbf{Passo 1:} Escrever a transformação e a inversa

A transformação é \[
Y = g(X) = \frac{X - \mu}{\sigma}.
\]

Isolando \(X\) em função de \(Y\), obtemos a inversa: \[
X = \sigma Y + \mu.
\]

Isto é, \[
x = g^{-1}(y) = \sigma y + \mu.
\]
\end{frame}

\begin{frame}{Função de Variável Aleatória: Caso Contínuo}
\phantomsection\label{funuxe7uxe3o-de-variuxe1vel-aleatuxf3ria-caso-contuxednuo-12}
\textbf{Passo 2:} Calcular o jacobiano em 1D

No caso unidimensional, o jacobiano é o valor absoluto da derivada de
\(x\) em relação a \(y\): \[
\dfrac{d}{dy} x= \dfrac{d}{dy} \,\sigma y + \mu  =  \sigma \quad \Rightarrow \quad \left| \dfrac{d}{dy} \,\sigma y + \mu \right| = \sigma
\]
\end{frame}

\begin{frame}{Função de Variável Aleatória: Caso Contínuo}
\phantomsection\label{funuxe7uxe3o-de-variuxe1vel-aleatuxf3ria-caso-contuxednuo-13}
\textbf{Passo 3:} Para \(Y = g(X)\) com \(g\) monotônica, a densidade de
\(Y\) é dada por

\[
f_Y(y) = f_X\big(g^{-1}(y)\big)\,\left|\frac{d}{dy} g^{-1}(y)\right|
\]

No nosso caso:

\[
\begin{aligned}
f_Y(y)
 &= f_X(\sigma y + \mu)\,\sigma \\[6pt]
 &= \frac{1}{\sigma\sqrt{2\pi}}
    \exp\!\left(
      -\frac{(\sigma y + \mu - \mu)^2}{2\sigma^2}
    \right)\,\sigma \\[6pt]
 &= \frac{1}{\sigma\sqrt{2\pi}}
    \exp\!\left(
      -\frac{(\sigma y)^2}{2\sigma^2}
    \right)\,\sigma = \frac{1}{\sqrt{2\pi}}
    \exp\!\left( -\frac{y^2}{2} \right).
\end{aligned}
\]
\end{frame}

\begin{frame}{Função de Variável Aleatória: Caso Contínuo}
\phantomsection\label{funuxe7uxe3o-de-variuxe1vel-aleatuxf3ria-caso-contuxednuo-14}
Logo, \[
f_Y(y) = \sigma \cdot \dfrac{1}{\sigma \sqrt{2\pi}}
\exp\left( -\dfrac{\sigma^2 y^2}{2\sigma^2} \right)
= \dfrac{1}{\sqrt{2\pi}}
\exp\left( -\dfrac{y^2}{2} \right), \quad y \in \mathbb{R}
\]

que é exatamente a densidade da \textbf{Normal Padrão}.

Portanto, \[
Y = \frac{X - \mu}{\sigma} \sim N(0,1).
\]
\end{frame}

\begin{frame}{Função de Variável Aleatória: Caso Contínuo}
\phantomsection\label{funuxe7uxe3o-de-variuxe1vel-aleatuxf3ria-caso-contuxednuo-15}
\textbf{Exemplo 08:} Seja \(X \sim U(0,1)\). Obtenha a função densidade
de \(Y = -\ln(X)\).

\pause

\textbf{Solução:}

Sabemos que se \(X \sim U(0,1)\), então sua função densidade \(f_X(x)\)
é dada por

\[
f_X(x) =
\begin{cases}
1, & 0 < x < 1, \\
0, & \text{caso contrário.}
\end{cases}
\]

Temos que \(Y = -\ln(X)\). Como \(0 < X < 1\), então \(Y > 0\).

Invertendo:

\[
Y = -\ln(X) \;\Rightarrow\; -Y = \ln(X) \;\Rightarrow\; X = e^{-Y}
\]
\end{frame}

\begin{frame}{Função de Variável Aleatória: Caso Contínuo}
\phantomsection\label{funuxe7uxe3o-de-variuxe1vel-aleatuxf3ria-caso-contuxednuo-16}
Portanto:

\[
x = g^{-1}(y) = e^{-y}
\]

\textbf{Jacobiano}

\[
\dfrac{d}{dy} x = \dfrac{d}{dy} e^{-y} = -e^{-y} \quad \Rightarrow \quad \left| \dfrac{d}{dy} x \right| = e^{-y}
\]

Temos que

\[
f_Y(y) = f_X(g^{-1}(y)) \left| \frac{d}{dy} g^{-1}(y) \right|
\]
\end{frame}

\begin{frame}{Função de Variável Aleatória: Caso Contínuo}
\phantomsection\label{funuxe7uxe3o-de-variuxe1vel-aleatuxf3ria-caso-contuxednuo-17}
Substituindo:

\[
f_Y(y) = f_X(e^{-y}) \cdot e^{-y}
\]

uma vez que \(f_X(x) = 1\) no intervalo \((0, 1)\).

\[
f_Y(y) = 1 \cdot e^{-y} = e^{-y}, \quad y > 0.
\]

Temos que \(Y \sim Exp(\lambda = 1)\)
\end{frame}

\begin{frame}{Função de Variável Aleatória: Caso Contínuo}
\phantomsection\label{funuxe7uxe3o-de-variuxe1vel-aleatuxf3ria-caso-contuxednuo-18}
\begin{block}{Generalização do Método do Jacobiano}
\phantomsection\label{generalizauxe7uxe3o-do-muxe9todo-do-jacobiano}
Se \(Y = g(X)\) não é monótona em \(\mathbb{R}\), podemos aplicar o
método do Jacobiano em cada um dos intervalos em que \(g\) é monótona,
da seguinte forma:

\begin{enumerate}
[1)]
\item
  Defina uma partição de \(\mathbb{R}_X\) formada pelos intervalos
  \(I_1, I_2, \cdots I_k\) tais que a função \(g\) é monótona em cada
  \(I_j, \,\,\, j= 1\cdots k\).
\item
  Obtenha
  \(f_j(y) = f_X \Big(g^{-1}(y)\Big) \Bigg|\dfrac{d}{dy}g^{-1}(y)\Bigg|\)
  a cada intervalo \(I_j\).
\item
  Finalmente, obtenha
\end{enumerate}

\[f_Y(y) = \sum_{j=1}^k f_j(y)\]
\end{block}
\end{frame}

\begin{frame}{Função de Variável Aleatória: Caso Contínuo}
\phantomsection\label{funuxe7uxe3o-de-variuxe1vel-aleatuxf3ria-caso-contuxednuo-19}
\textbf{Exemplo 09:} \textbf{(Exemplo 06 revisitado)} Seja \(X\) uma
variável aleatória tal que \(X \sim N(0,1)\), cuja \emph{fdp} é dada por

\[f_X(x) = \dfrac{1}{\sqrt{2\pi}}
\exp\left( -\dfrac{x^2}{2} \right), \quad x \in \mathbb{R}\]

Considerando \(Y = g(X) = X^2\), obter a função densidade de
probabilidade de \(Y\), usando o \textbf{método do Jacobiano}.
\end{frame}

\begin{frame}{Função de Variável Aleatória: Caso Contínuo}
\phantomsection\label{funuxe7uxe3o-de-variuxe1vel-aleatuxf3ria-caso-contuxednuo-20}
\textbf{Solução:}

Note que, neste caso, temos duas regiões disjuntas do suporte de \(X\),
\(S_X\), em que \(g(x) = x^2\) é injetora. Temos,
\(S_{X_1} = (-\infty, 0]\) e \(S_{X_2} = [0, \infty)\). Logo,
\(g^{-1}_1(y) = -\sqrt{y}\) para \(x \le 0\) e
\(g^{-1}_2(y) = \sqrt{y}\) para \(x > 0\). Portanto,
\end{frame}

\begin{frame}{Função de Variável Aleatória: Caso Contínuo}
\phantomsection\label{funuxe7uxe3o-de-variuxe1vel-aleatuxf3ria-caso-contuxednuo-21}
\[
\begin{aligned}
f_Y(y)
 &= f_X(g_1^{-1}(y)) \left| \frac{d}{dy} g_1^{-1}(y) \right|
  + f_X(g_2^{-1}(y)) \left| \frac{d}{dy} g_2^{-1}(y) \right| \\[6pt]
 &= \frac{1}{\sqrt{2\pi}}
    \exp\!\left( -\frac{(-\sqrt{y})^2}{2} \right)
    \left| \frac{d}{dy} (-\sqrt{y}) \right|
  + \frac{1}{\sqrt{2\pi}}
    \exp\!\left( -\frac{(\sqrt{y})^2}{2} \right)
    \left| \frac{d}{dy} \sqrt{y} \right| \\[6pt]
 &= \frac{1}{\sqrt{2\pi}}
    \exp\!\left( -\frac{y}{2} \right)
    \left| -\frac{1}{2 \sqrt{y}} \right|
  + \frac{1}{\sqrt{2\pi}}
    \exp\!\left( -\frac{y}{2} \right)
    \left| \frac{1}{2 \sqrt{y}}  \right| \\[6pt]
 &= \frac{1}{\sqrt{2\pi}}
    e^{-y/2}\,\frac{1}{2 \sqrt{y}}
  + \frac{1}{\sqrt{2\pi}}
    e^{-y/2}\,\frac{1}{2 \sqrt{y}} \\[6pt]
 &= \frac{2}{\sqrt{2\pi}}\,e^{-y/2}\,\frac{1}{2 \sqrt{y}} = \frac{1}{\sqrt{2\pi}}\,e^{-y/2}\,y^{-1/2}
  = \frac{1}{2^{1/2}\sqrt{\pi}}\,e^{-y/2}\,y^{-1/2}
  = \chi^2_1(y).
\end{aligned}
\]
\end{frame}

\begin{frame}{Função de Variável Aleatória: Caso Contínuo}
\phantomsection\label{funuxe7uxe3o-de-variuxe1vel-aleatuxf3ria-caso-contuxednuo-22}
\textbf{Exemplo 10:} Seja \(X\sim U(-1,1)\). Qual a distribuição de
\(Y=X^2\)?

\pause

\textbf{Solução:}

Definindo a função indicadora de um conjunto \(A\) como

\[
I_A(x) =
\begin{cases}
1, & \text{se } x \in A, \\
0, & \text{se } x \notin A
\end{cases}
\]

podemos escrever a fução densidade de \(X\) como
\(f_X(x) = \dfrac{1}{2}I_{(-1,1)}(x)\).
\end{frame}

\begin{frame}{Função de Variável Aleatória: Caso Contínuo}
\phantomsection\label{funuxe7uxe3o-de-variuxe1vel-aleatuxf3ria-caso-contuxednuo-23}
Como \(Y = X^2\) não é monótona em \((−1, 1)\), mas o é em \((−1, 0)\) e
\((0, 1)\) podemos utilizar a generalização do método do Jacobiano, ou
seja, no intervalo \((−1, 0)\), \(g_1^{-1}(y) = -\sqrt{y}\) e no
intervalo \((0, 1)\) \(g_2^{-1}(y) = \sqrt{y}\). Assim,
\end{frame}

\begin{frame}{Função de Variável Aleatória: Caso Contínuo}
\phantomsection\label{funuxe7uxe3o-de-variuxe1vel-aleatuxf3ria-caso-contuxednuo-24}
\[
\begin{aligned}
f_Y(y)
 &= f_X(g_1^{-1}(y)) \left| \frac{d}{dy} g_1^{-1}(y) \right|
  + f_X(g_2^{-1}(y)) \left| \frac{d}{dy} g_2^{-1}(y) \right| \\[6pt]
 &= f_X(-\sqrt{y}) \left| \frac{d}{dy} (-\sqrt{y}) \right|
  + f_X(\sqrt{y}) \left| \frac{d}{dy} \sqrt{y} \right| \\[6pt]
 &= \frac{1}{2} \cdot \left| -\frac{1}{2 \sqrt{y}} \right|
  + \frac{1}{2} \cdot \left| \frac{1}{2 \sqrt{y}} \right| \\[6pt]
 &= \frac{1}{2} \cdot \frac{1}{2 \sqrt{y}}
  + \frac{1}{2} \cdot \frac{1}{2 \sqrt{y}} \\[6pt]
 &= \frac{1}{4 \sqrt{y}} + \frac{1}{4 \sqrt{y}}
  = \frac{1}{2 \sqrt{y}}, \qquad y > 0
\end{aligned}
\]
\end{frame}

\begin{frame}{Função de Variável Aleatória: Caso Contínuo}
\phantomsection\label{funuxe7uxe3o-de-variuxe1vel-aleatuxf3ria-caso-contuxednuo-25}
Logo, se \(X\sim U(-1,1)\), a distribuição de \(Y=X^2\) é dada por

\[f_Y(y) = \dfrac{1}{2 \sqrt{y}}I_{(0,1)}(y)\]
\end{frame}

\begin{frame}{A Transformação Integral}
\phantomsection\label{a-transformauxe7uxe3o-integral}
Um importante e conhecido teorema é o \textbf{Teorema da Probabilidade
Integral}, que relaciona a distribuição uniforme contínua com todas as
outras distribuições de probabilidade.

\pause

Seu resultado é importante na \textbf{Estatística Computacional}, pois
permite que sejam simuladas amostras aleatórias de qualquer distribuição
de probabilidade.
\end{frame}

\begin{frame}{A Transformação Integral}
\phantomsection\label{a-transformauxe7uxe3o-integral-1}
\textbf{Teorema da probabilidade integral:} Consideremos \(X\) uma
variável aleatória absolutamente contínua com função de distribuição
acumulada \(F_X\). Então \(U = F_X(X)\) possui distribuição uniforme no
intervalo \([0,1]\). Por outro lado, se \(U \sim U(0,1)\), então
\(X = F^{-1}_X(U)\) possui função de distribuição acumulada \(F_X\).
\end{frame}

\begin{frame}{A Transformação Integral}
\phantomsection\label{a-transformauxe7uxe3o-integral-2}
\textbf{Demonstração:} Considere \(X\) uma variável aleatória
absolutamente contínua com função de distribuição acumulada \(F_X\),
então

\[
F_U(u) = P(U \le u)= P(F_X(X) \le u) = P(X \le F_X^{-1}(u)) = F_X(F_X^{-1}(u)) = u
\] Por outro lado, se \(U \sim U(0,1)\), então

\[
{\small  
P(X \le x) = P(F^{-1}_X(U) \le x) = P(U \le F_x(x)) = F_U(F_X(x)) = F_X(x), \,\, \text{pois } U \text{ é uniforme} 
}
\] Assim \(X\) possui função de distribuição \(F_X\).
\end{frame}

\begin{frame}{A Transformação Integral}
\phantomsection\label{a-transformauxe7uxe3o-integral-3}
\textbf{Exemplo 11:} \textbf{(Geração de variáveis da distribuição
exponencial)} Considere que \(X \sim Exp(\lambda)\), cuja função de
distribuição de probabilidade é dada por

\[ F_X(x) = 1 - e^{-\lambda x}, \qquad x \ge 0 \]

Se \(U \sim U(0,1)\), qual é a distribuição de \(X = F_X^{-1}(U)\)?

\pause

\textbf{Solução:}

A função inversa é dada por

\[
u = 1 - e^{-\lambda x} \Rightarrow e^{-\lambda x} = 1 - u \Rightarrow x = -\frac{1}{\lambda} \ln(1-u)
\]
\end{frame}

\begin{frame}{A Transformação Integral}
\phantomsection\label{a-transformauxe7uxe3o-integral-4}
Assim, pelo teorema da transformação integral,

\[X = F_X^{-1}(U) = -\frac{1}{\lambda} \ln(1-U)\] possui distribuição
exponencial com função de distribuição \(F_X\). Podemos utilizar esse
resultado para gerar realizações de uma amostra aleatória da
distribuição exponencial. Para isso, basta gerarmos realizações
uniformes \(u\) e aplicarmos a relação anterior.
\end{frame}

\begin{frame}{A Transformação Integral}
\phantomsection\label{a-transformauxe7uxe3o-integral-5}
\textbf{Exemplo 12:} \textbf{(Geração de variáveis da distribuição
Weibull)} Considere que \(X \sim Weibull(\alpha, \beta)\), cuja função
de distribuição de probabilidade é dada por

\[F_X(x) = 1 - e^{-(x/\alpha)^{\beta}}, \qquad x \ge 0\]

Se \(U \sim U(0,1)\), qual é a distribuição de \(X = F_X^{-1}(U)\)?

\pause

\textbf{Solução:}

A função inversa é dada por

\[
u = 1 - e^{-(x/\alpha)^{\beta}} \Rightarrow e^{-(x/\alpha)^{\beta}} = 1 - u \Rightarrow (x/\alpha)^{\beta} = -\ln(1-u) \Rightarrow x = \alpha \,[-\ln(1-u)]^{1/\beta}
\]
\end{frame}

\begin{frame}{A Transformação Integral}
\phantomsection\label{a-transformauxe7uxe3o-integral-6}
Assim, pelo teorema da transformação integral,

\[X = F_X^{-1}(U) = \alpha \,[-\ln(1-U)]^{1/\beta}\] possui distribuição
Weibull com função de distribuição \(F_X\).
\end{frame}




\end{document}

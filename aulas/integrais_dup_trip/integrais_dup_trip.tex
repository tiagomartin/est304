% Options for packages loaded elsewhere
% Options for packages loaded elsewhere
\PassOptionsToPackage{unicode}{hyperref}
\PassOptionsToPackage{hyphens}{url}
%
\documentclass[
  ignorenonframetext,
  aspectratio=169,
]{beamer}
\newif\ifbibliography
\usepackage{pgfpages}
\setbeamertemplate{caption}[numbered]
\setbeamertemplate{caption label separator}{: }
\setbeamercolor{caption name}{fg=normal text.fg}
\beamertemplatenavigationsymbolsempty
% remove section numbering
\setbeamertemplate{part page}{
  \centering
  \begin{beamercolorbox}[sep=16pt,center]{part title}
    \usebeamerfont{part title}\insertpart\par
  \end{beamercolorbox}
}
\setbeamertemplate{section page}{
  \centering
  \begin{beamercolorbox}[sep=12pt,center]{section title}
    \usebeamerfont{section title}\insertsection\par
  \end{beamercolorbox}
}
\setbeamertemplate{subsection page}{
  \centering
  \begin{beamercolorbox}[sep=8pt,center]{subsection title}
    \usebeamerfont{subsection title}\insertsubsection\par
  \end{beamercolorbox}
}
% Prevent slide breaks in the middle of a paragraph
\widowpenalties 1 10000
\raggedbottom
\AtBeginPart{
  \frame{\partpage}
}
\AtBeginSection{
  \ifbibliography
  \else
    \frame{\sectionpage}
  \fi
}
\AtBeginSubsection{
  \frame{\subsectionpage}
}
\usepackage{iftex}
\ifPDFTeX
  \usepackage[T1]{fontenc}
  \usepackage[utf8]{inputenc}
  \usepackage{textcomp} % provide euro and other symbols
\else % if luatex or xetex
  \usepackage{unicode-math} % this also loads fontspec
  \defaultfontfeatures{Scale=MatchLowercase}
  \defaultfontfeatures[\rmfamily]{Ligatures=TeX,Scale=1}
\fi
\usepackage{lmodern}

\ifPDFTeX\else
  % xetex/luatex font selection
\fi
% Use upquote if available, for straight quotes in verbatim environments
\IfFileExists{upquote.sty}{\usepackage{upquote}}{}
\IfFileExists{microtype.sty}{% use microtype if available
  \usepackage[]{microtype}
  \UseMicrotypeSet[protrusion]{basicmath} % disable protrusion for tt fonts
}{}
\makeatletter
\@ifundefined{KOMAClassName}{% if non-KOMA class
  \IfFileExists{parskip.sty}{%
    \usepackage{parskip}
  }{% else
    \setlength{\parindent}{0pt}
    \setlength{\parskip}{6pt plus 2pt minus 1pt}}
}{% if KOMA class
  \KOMAoptions{parskip=half}}
\makeatother


\usepackage{longtable,booktabs,array}
\usepackage{calc} % for calculating minipage widths
\usepackage{caption}
% Make caption package work with longtable
\makeatletter
\def\fnum@table{\tablename~\thetable}
\makeatother
\usepackage{graphicx}
\makeatletter
\newsavebox\pandoc@box
\newcommand*\pandocbounded[1]{% scales image to fit in text height/width
  \sbox\pandoc@box{#1}%
  \Gscale@div\@tempa{\textheight}{\dimexpr\ht\pandoc@box+\dp\pandoc@box\relax}%
  \Gscale@div\@tempb{\linewidth}{\wd\pandoc@box}%
  \ifdim\@tempb\p@<\@tempa\p@\let\@tempa\@tempb\fi% select the smaller of both
  \ifdim\@tempa\p@<\p@\scalebox{\@tempa}{\usebox\pandoc@box}%
  \else\usebox{\pandoc@box}%
  \fi%
}
% Set default figure placement to htbp
\def\fps@figure{htbp}
\makeatother





\setlength{\emergencystretch}{3em} % prevent overfull lines

\providecommand{\tightlist}{%
  \setlength{\itemsep}{0pt}\setlength{\parskip}{0pt}}



 


\makeatletter
\@ifpackageloaded{caption}{}{\usepackage{caption}}
\AtBeginDocument{%
\ifdefined\contentsname
  \renewcommand*\contentsname{Table of contents}
\else
  \newcommand\contentsname{Table of contents}
\fi
\ifdefined\listfigurename
  \renewcommand*\listfigurename{List of Figures}
\else
  \newcommand\listfigurename{List of Figures}
\fi
\ifdefined\listtablename
  \renewcommand*\listtablename{List of Tables}
\else
  \newcommand\listtablename{List of Tables}
\fi
\ifdefined\figurename
  \renewcommand*\figurename{Figure}
\else
  \newcommand\figurename{Figure}
\fi
\ifdefined\tablename
  \renewcommand*\tablename{Table}
\else
  \newcommand\tablename{Table}
\fi
}
\@ifpackageloaded{float}{}{\usepackage{float}}
\floatstyle{ruled}
\@ifundefined{c@chapter}{\newfloat{codelisting}{h}{lop}}{\newfloat{codelisting}{h}{lop}[chapter]}
\floatname{codelisting}{Listing}
\newcommand*\listoflistings{\listof{codelisting}{List of Listings}}
\makeatother
\makeatletter
\makeatother
\makeatletter
\@ifpackageloaded{caption}{}{\usepackage{caption}}
\@ifpackageloaded{subcaption}{}{\usepackage{subcaption}}
\makeatother

\usepackage{bookmark}
\IfFileExists{xurl.sty}{\usepackage{xurl}}{} % add URL line breaks if available
\urlstyle{same}
\hypersetup{
  pdftitle={Integrais Duplas e Triplas},
  hidelinks,
  pdfcreator={LaTeX via pandoc}}



\title{Integrais Duplas e Triplas}
\author{}
\date{}

\begin{document}
\frame{\titlepage}


\begin{frame}{Introdução}
\phantomsection\label{introduuxe7uxe3o}
Na teoria da probabilidade, especialmente quando lidamos com
\textbf{vetores aleatórios contínuos}, as \textbf{integrais múltiplas}
desempenham um papel central. Diferentemente do caso discreto, em que
probabilidades são obtidas por somas, no caso contínuo as probabilidades
são calculadas por integrais, que admitem uma interpretação geométrica
natural como volumes sob superfícies.
\end{frame}

\begin{frame}{Integrais Duplas}
\phantomsection\label{integrais-duplas}
Seja \(f:\mathbb{R}^2 \to \mathbb{R}\) uma função não negativa e
integrável em uma região \(R \subset \mathbb{R}^2\). A integral dupla de
\(f\) sobre \(R\) é definida por

\[
\iint_R f(x,y)\,dA
\]

Geometricamente, essa integral representa o volume do sólido limitado
superiormente pela superfície \(z=f(x,y)\), inferiormente pelo plano
\(z=0\) e lateralmente pela região \(R\).
\end{frame}

\begin{frame}{Integrais Duplas}
\phantomsection\label{integrais-duplas-1}
\begin{center}
\pandocbounded{\includegraphics[keepaspectratio]{../../images/repres_int_dupla.png}}
\end{center}

Aqui, \(R\) é a região sobre a qual integramos e \(dA\) é o elemento de
área, uma versão infinitesimal de \(\delta A\).
\end{frame}

\begin{frame}{Integrais Duplas}
\phantomsection\label{integrais-duplas-2}
\begin{itemize}
\tightlist
\item
  \(dA\) depende das coordenadas. Em coordenadas cartesianas, temos:
\end{itemize}

\begin{center}
\pandocbounded{\includegraphics[keepaspectratio]{../../images/regiao_dA.png}}
\end{center}

Vemos claramente que \(\delta A = \delta x \delta y\), então, temos
\(dA=dxdy\).
\end{frame}

\begin{frame}{Integrais Duplas}
\phantomsection\label{integrais-duplas-3}
Por conseguinte, esta integral é avaliada como:

\[
\iint_R f(x,y)\,dA = \iint_R f(x,y)\,dxdy
\]

\pause

Na prática, integrais duplas são calculadas como \textbf{integrais
iteradas}. Seja

\[
R=\{(x,y): a\le x\le b,\ c\le y\le d\}
\]

Assim, pelo \textbf{Teorema de Fubini}, a integral é:

\[
\iint_R f(x,y)\,dx\,dy
=
\int_{y=c}^{d}
\left(
\int_{x=a}^{b} f(x,y)\,dx
\right)
dy
=
\int_{x=a}^{b}
\left(
\int_{y=c}^{d} f(x,y)\,dy
\right)
dx
\]
\end{frame}

\begin{frame}{Integrais Duplas}
\phantomsection\label{integrais-duplas-4}
Primeiro, calculamos a integral interna e depois a integral externa.
Neste caso, a integral interna é em relação a \(x\). No entanto, a ordem
de integração não importa, pois podemos calcular a integral em relação a
\(y\) primeiro.
\end{frame}

\begin{frame}{Integrais Duplas}
\phantomsection\label{integrais-duplas-5}
\textbf{Exemplo 01:} Calcule

\[
\iint_{[0,1]\times[0,1]} (x+y)\,dx\,dy
\]

\pause

\textbf{Solução:} Como \([0,1]\times[0,1]\) é um retângulo, podemos
escrever a integral dupla como uma integral iterada:

\[
\iint_{[0,1]\times[0,1]} (x+y)\,dx\,dy
=
\int_{0}^{1}\int_{0}^{1} (x+y)\,dx\,dy
\]
\end{frame}

\begin{frame}{Integrais Duplas}
\phantomsection\label{integrais-duplas-6}
\textbf{Passo 1:} integrar em relação a \(x\)

Aqui \(y\) é constante. Então:

\[
\int_{0}^{1} (x+y)\,dx
=
\int_{0}^{1} x\,dx + \int_{0}^{1} y\,dx
\]

Calculando cada parte:

\[
\int_{0}^{1} x\,dx = \left.\frac{x^2}{2}\right|_{0}^{1}=\frac12,
\qquad
\int_{0}^{1} y\,dx = y\int_{0}^{1} 1\,dx = y(1-0)=y
\]
\end{frame}

\begin{frame}{Integrais Duplas}
\phantomsection\label{integrais-duplas-7}
Logo, a integral interna é: \[
\int_{0}^{1} (x+y)\,dx = \frac12 + y
\]

\pause

\textbf{Passo 2:} integrar em relação a \(y\)

Agora integramos o resultado de \(0\) a \(1\):

\[
\int_{0}^{1}\left(\frac12+y\right)\,dy
=
\int_{0}^{1}\frac12\,dy+\int_{0}^{1}y\,dy
\]
\end{frame}

\begin{frame}{Integrais Duplas}
\phantomsection\label{integrais-duplas-8}
\[
\int_{0}^{1}\frac12\,dy=\left.\frac{y}{2}\right|_{0}^{1}=\frac12,
\qquad
\int_{0}^{1}y\,dy=\left.\frac{y^2}{2}\right|_{0}^{1}=\frac12
\]

Somando: \[
\int_{0}^{1}\left(\frac12+y\right)\,dy=\frac12+\frac12=1
\]

Logo,

\[
\iint_{[0,1]\times[0,1]} (x+y)\,dx\,dy = 1
\]
\end{frame}

\begin{frame}{Integrais Duplas}
\phantomsection\label{integrais-duplas-9}
\textbf{Exemplo 02:} Na integral dupla seguinte, indicar a região \(R\)
de integração e encontrar o seu valor:

\[
\int_{0}^{2}\int_{0}^{1} (1 + 2x + 2y)\,dy\,dx
\]

\pause

\textbf{Solução:} A integral está escrita na forma iterada

\[
\int_{x=0}^{2}\int_{y=0}^{1} (\cdot)\,dy\,dx
\]

Logo, a região de integração é

\[
R = \{(x,y)\in\mathbb{R}^2 : 0 \le x \le 2,\ 0 \le y \le 1\}
\]
\end{frame}

\begin{frame}{Integrais Duplas}
\phantomsection\label{integrais-duplas-10}
Trata-se de um \textbf{retângulo} no plano cartesiano, com base no
intervalo \([0,2]\) do eixo \(x\) e altura no intervalo \([0,1]\) do
eixo \(y\). Assim, mantendo \(x\) fixo, temos:

\[
\begin{aligned}
\int_{0}^{1} (1 + 2x + 2y)\,dy &= \int_{0}^{1} 1\,dy + \int_{0}^{1} 2x\,dy + \int_{0}^{1} 2y\,dy \\ & = y \bigg|_0^1 + 2x\,y \bigg|_0^1 + 2\dfrac{y^2}{2}\bigg|_0^1 \\ &= 1 + 2x + 1 \\ & = 2 + 2x
\end{aligned}
\]
\end{frame}

\begin{frame}{Integrais Duplas}
\phantomsection\label{integrais-duplas-11}
vamos agora, integrar em relação a \(x\):

\[
\begin{aligned}
\int_{0}^{2} (2 + 2x)\,dx &= \int_{0}^{2} 2\,dx + \int_{0}^{2} 2x\,dx \\ & = 2x \bigg|_0^2 + 2\dfrac{x^2}{2}\bigg|_0^2 \\ &= 4 + 4 \\ & = 8
\end{aligned}
\]
\end{frame}

\begin{frame}{Integrais Duplas}
\phantomsection\label{integrais-duplas-12}
Esse valor representa o \textbf{volume} do sólido limitado:

\begin{itemize}
\tightlist
\item
  inferiormente pelo plano \(z=0\),
\item
  superiormente pela superfície \(z=1+2x+2y\),
\item
  lateralmente pela região retangular \(R=[0,2]\times[0,1]\)
\end{itemize}
\end{frame}

\begin{frame}{Integrais Duplas em Regiões Gerais}
\phantomsection\label{integrais-duplas-em-regiuxf5es-gerais}
Até agora, consideramos integrais duplas sobre regiões retangulares. No
entanto, em muitas aplicações, especialmente em Probabilidade, a região
de integração não é um retângulo, mas sim uma região mais geral do
plano.

\pause

Nesses casos, a integral dupla continua sendo definida como

\[
\iint_R f(x,y)\,dx\,dy,
\]

mas o cálculo exige uma descrição adequada da região \(R\).
\end{frame}

\begin{frame}{Regiões Gerais no Plano}
\phantomsection\label{regiuxf5es-gerais-no-plano}
Uma \textbf{região geral} \(R \subset \mathbb{R}^2\) é qualquer
subconjunto do plano que possa ser descrito por desigualdades envolvendo
\(x\) e \(y\).

\pause

Na prática, trabalhamos principalmente com dois tipos de regiões:

\begin{itemize}
\tightlist
\item
  Regiões do tipo I (ou regiões \(x\))
\item
  Regiões do tipo II (ou regiões \(y\))
\end{itemize}
\end{frame}

\begin{frame}{Regiões Gerais no Plano}
\phantomsection\label{regiuxf5es-gerais-no-plano-1}
\begin{center}
\pandocbounded{\includegraphics[keepaspectratio]{../../images/regioes_gerais.png}}
\end{center}
\end{frame}

\begin{frame}{Regiões do Tipo I (Regiões \(x\))}
\phantomsection\label{regiuxf5es-do-tipo-i-regiuxf5es-x}
Uma região \(R\) é dita do tipo I se pode ser escrita na forma

\[
R = \{(x,y) : a \le x \le b,\ g_1(x) \le y \le g_2(x)\}
\]

Nesse caso, a integral dupla é calculada como

\[
\iint_R f(x,y)\,dx\,dy
=
\int_a^b \int_{g_1(x)}^{g_2(x)} f(x,y)\,dy\,dx
\]

\begin{block}{Interpretação geométrica}
\phantomsection\label{interpretauxe7uxe3o-geomuxe9trica}
\begin{itemize}
\tightlist
\item
  \(x\) varia em um intervalo fixo \([a,b]\);
\item
  para cada \(x\), a variável \(y\) varia entre duas curvas.
\end{itemize}
\end{block}
\end{frame}

\begin{frame}{Exemplo: Região do Tipo I}
\phantomsection\label{exemplo-regiuxe3o-do-tipo-i}
Considere a região limitada pelas curvas

\[
y = x^2 \quad \text{e} \quad y = x,
\] com \(0 \le x \le 1\).

A região é

\[
R = \{(x,y): 0 \le x \le 1,\ x^2 \le y \le x\}
\]

A integral dupla de \(f(x,y)\) sobre \(R\) é

\[
\iint_R f(x,y)\,dx\,dy
=
\int_0^1 \int_{x^2}^{x} f(x,y)\,dy\,dx
\]
\end{frame}

\begin{frame}{Exemplo: Região do Tipo I}
\phantomsection\label{exemplo-regiuxe3o-do-tipo-i-1}
\pandocbounded{\includegraphics[keepaspectratio]{integrais_dup_trip_files/figure-beamer/unnamed-chunk-1-1.pdf}}
\end{frame}

\begin{frame}{Regiões do Tipo II (Regiões \(y\))}
\phantomsection\label{regiuxf5es-do-tipo-ii-regiuxf5es-y}
Uma região \(R\) é dita do tipo II se pode ser escrita como

\[
R = \{(x,y) : c \le y \le d,\ h_1(y) \le x \le h_2(y)\}
\]

Nesse caso,

\[
\iint_R f(x,y)\,dx\,dy
=
\int_c^d \int_{h_1(y)}^{h_2(y)} f(x,y)\,dx\,dy
\]

\begin{block}{Interpretação geométrica}
\phantomsection\label{interpretauxe7uxe3o-geomuxe9trica-1}
\begin{itemize}
\tightlist
\item
  \(y\) varia em um intervalo fixo \([c,d]\);
\item
  para cada \(y\), a variável \(x\) varia entre duas curvas.
\end{itemize}
\end{block}
\end{frame}

\begin{frame}{Exemplo: Região do Tipo II}
\phantomsection\label{exemplo-regiuxe3o-do-tipo-ii}
Considere a região limitada por

\[
x = y^2 \quad \text{e} \quad x = y,
\] com \(0 \le y \le 1\).

A região é

\[
R = \{(x,y): 0 \le y \le 1,\ y^2 \le x \le y\}
\]

A integral dupla de \(f(x,y)\) sobre \(R\) é \[
\iint_R f(x,y)\,dx\,dy
=
\int_0^1 \int_{y^2}^{y} f(x,y)\,dx\,dy
\]
\end{frame}

\begin{frame}{Exemplo: Região do Tipo II}
\phantomsection\label{exemplo-regiuxe3o-do-tipo-ii-1}
\pandocbounded{\includegraphics[keepaspectratio]{integrais_dup_trip_files/figure-beamer/unnamed-chunk-2-1.pdf}}
\end{frame}

\begin{frame}{Troca da Ordem de Integração}
\phantomsection\label{troca-da-ordem-de-integrauxe7uxe3o}
Uma mesma região \(R\) pode, muitas vezes, ser descrita tanto como
região do tipo I quanto do tipo II.

Trocar a ordem de integração significa reescrever

\[
\int_a^b \int_{g_1(x)}^{g_2(x)} f(x,y)\,dy\,dx
\]

na forma

\[
\int_c^d \int_{h_1(y)}^{h_2(y)} f(x,y)\,dx\,dy,
\]

mantendo a mesma região \(R\).
\end{frame}

\begin{frame}{Troca da Ordem de Integração}
\phantomsection\label{troca-da-ordem-de-integrauxe7uxe3o-1}
\pandocbounded{\includegraphics[keepaspectratio]{integrais_dup_trip_files/figure-beamer/unnamed-chunk-3-1.pdf}}
\end{frame}




\end{document}

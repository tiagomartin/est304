% Options for packages loaded elsewhere
% Options for packages loaded elsewhere
\PassOptionsToPackage{unicode}{hyperref}
\PassOptionsToPackage{hyphens}{url}
%
\documentclass[
  ignorenonframetext,
  aspectratio=169,
]{beamer}
\newif\ifbibliography
\usepackage{pgfpages}
\setbeamertemplate{caption}[numbered]
\setbeamertemplate{caption label separator}{: }
\setbeamercolor{caption name}{fg=normal text.fg}
\beamertemplatenavigationsymbolsempty
% remove section numbering
\setbeamertemplate{part page}{
  \centering
  \begin{beamercolorbox}[sep=16pt,center]{part title}
    \usebeamerfont{part title}\insertpart\par
  \end{beamercolorbox}
}
\setbeamertemplate{section page}{
  \centering
  \begin{beamercolorbox}[sep=12pt,center]{section title}
    \usebeamerfont{section title}\insertsection\par
  \end{beamercolorbox}
}
\setbeamertemplate{subsection page}{
  \centering
  \begin{beamercolorbox}[sep=8pt,center]{subsection title}
    \usebeamerfont{subsection title}\insertsubsection\par
  \end{beamercolorbox}
}
% Prevent slide breaks in the middle of a paragraph
\widowpenalties 1 10000
\raggedbottom
\AtBeginPart{
  \frame{\partpage}
}
\AtBeginSection{
  \ifbibliography
  \else
    \frame{\sectionpage}
  \fi
}
\AtBeginSubsection{
  \frame{\subsectionpage}
}
\usepackage{iftex}
\ifPDFTeX
  \usepackage[T1]{fontenc}
  \usepackage[utf8]{inputenc}
  \usepackage{textcomp} % provide euro and other symbols
\else % if luatex or xetex
  \usepackage{unicode-math} % this also loads fontspec
  \defaultfontfeatures{Scale=MatchLowercase}
  \defaultfontfeatures[\rmfamily]{Ligatures=TeX,Scale=1}
\fi
\usepackage{lmodern}

\ifPDFTeX\else
  % xetex/luatex font selection
\fi
% Use upquote if available, for straight quotes in verbatim environments
\IfFileExists{upquote.sty}{\usepackage{upquote}}{}
\IfFileExists{microtype.sty}{% use microtype if available
  \usepackage[]{microtype}
  \UseMicrotypeSet[protrusion]{basicmath} % disable protrusion for tt fonts
}{}
\makeatletter
\@ifundefined{KOMAClassName}{% if non-KOMA class
  \IfFileExists{parskip.sty}{%
    \usepackage{parskip}
  }{% else
    \setlength{\parindent}{0pt}
    \setlength{\parskip}{6pt plus 2pt minus 1pt}}
}{% if KOMA class
  \KOMAoptions{parskip=half}}
\makeatother


\usepackage{longtable,booktabs,array}
\usepackage{calc} % for calculating minipage widths
\usepackage{caption}
% Make caption package work with longtable
\makeatletter
\def\fnum@table{\tablename~\thetable}
\makeatother
\usepackage{graphicx}
\makeatletter
\newsavebox\pandoc@box
\newcommand*\pandocbounded[1]{% scales image to fit in text height/width
  \sbox\pandoc@box{#1}%
  \Gscale@div\@tempa{\textheight}{\dimexpr\ht\pandoc@box+\dp\pandoc@box\relax}%
  \Gscale@div\@tempb{\linewidth}{\wd\pandoc@box}%
  \ifdim\@tempb\p@<\@tempa\p@\let\@tempa\@tempb\fi% select the smaller of both
  \ifdim\@tempa\p@<\p@\scalebox{\@tempa}{\usebox\pandoc@box}%
  \else\usebox{\pandoc@box}%
  \fi%
}
% Set default figure placement to htbp
\def\fps@figure{htbp}
\makeatother





\setlength{\emergencystretch}{3em} % prevent overfull lines

\providecommand{\tightlist}{%
  \setlength{\itemsep}{0pt}\setlength{\parskip}{0pt}}



 


\makeatletter
\@ifpackageloaded{caption}{}{\usepackage{caption}}
\AtBeginDocument{%
\ifdefined\contentsname
  \renewcommand*\contentsname{Table of contents}
\else
  \newcommand\contentsname{Table of contents}
\fi
\ifdefined\listfigurename
  \renewcommand*\listfigurename{List of Figures}
\else
  \newcommand\listfigurename{List of Figures}
\fi
\ifdefined\listtablename
  \renewcommand*\listtablename{List of Tables}
\else
  \newcommand\listtablename{List of Tables}
\fi
\ifdefined\figurename
  \renewcommand*\figurename{Figure}
\else
  \newcommand\figurename{Figure}
\fi
\ifdefined\tablename
  \renewcommand*\tablename{Table}
\else
  \newcommand\tablename{Table}
\fi
}
\@ifpackageloaded{float}{}{\usepackage{float}}
\floatstyle{ruled}
\@ifundefined{c@chapter}{\newfloat{codelisting}{h}{lop}}{\newfloat{codelisting}{h}{lop}[chapter]}
\floatname{codelisting}{Listing}
\newcommand*\listoflistings{\listof{codelisting}{List of Listings}}
\makeatother
\makeatletter
\makeatother
\makeatletter
\@ifpackageloaded{caption}{}{\usepackage{caption}}
\@ifpackageloaded{subcaption}{}{\usepackage{subcaption}}
\makeatother

\usepackage{bookmark}
\IfFileExists{xurl.sty}{\usepackage{xurl}}{} % add URL line breaks if available
\urlstyle{same}
\hypersetup{
  pdftitle={Vetores Aleatórios},
  hidelinks,
  pdfcreator={LaTeX via pandoc}}



\title{Vetores Aleatórios}
\author{}
\date{}

\begin{document}
\frame{\titlepage}


\begin{frame}{Introdução}
\phantomsection\label{introduuxe7uxe3o}
Em muitas situações, é comum que um experimento aleatório gere mais de
uma variável de interesse e, quase sempre, o interesse estará em estudar
o comportamento simultâneo de 2 ou mais variáveis, em busca de relações,
associações.

Torna-se necessário, então, conhecer o comportamento probabilístico
conjunto de tais variáveis.
\end{frame}

\begin{frame}{Variáveis Aleatórias Multidimensionais}
\phantomsection\label{variuxe1veis-aleatuxf3rias-multidimensionais}
\textbf{Definição:} Sejam \(\varepsilon\) um experimento e \(S\) um
espaço amostral associado a \(\varepsilon\). Sejam \(X = X(s)\) e
\(Y = Y(s)\) duas funções, cada uma associando um número real a cada
resultado \(s \in S\). Denominaremos \((X, Y)\) uma \textbf{variável
aleatória bidimensional} (também chamada \textbf{vetor aleatório}).

\begin{center}
\pandocbounded{\includegraphics[keepaspectratio]{../../images/vetor_2D.png}}
\end{center}
\end{frame}

\begin{frame}{Variáveis Aleatórias Multidimensionais}
\phantomsection\label{variuxe1veis-aleatuxf3rias-multidimensionais-1}
Se \(X_1 = X_1(s)\), \(X_2 = X_2(s)\), \(\ldots\), \(X_n = X_n(s)\)
forem \(n\) funções, cada uma associando um número real a cada resultado
\(s \in S\), denominaremos \((X_1, \ldots, X_n)\) uma \textbf{variável
aleatória \(n\)-dimensional} (ou um \textbf{vetor aleatório
\(n\)-dimensional}).
\end{frame}

\section{Caso discreto}\label{caso-discreto}

\begin{frame}{Variáveis Aleatórias Multidimensionais Discretas}
\phantomsection\label{variuxe1veis-aleatuxf3rias-multidimensionais-discretas}
\textbf{Definição:} \((X, Y)\) será uma \textbf{variável aleatória
discreta bidimensional} se os valores possíveis de \((X, Y)\) forem
finitos ou infinitos numeráveis. Isto é, os valores possíveis de
\((X, Y)\) possam ser representados por \((x_i, y_j)\),
\(i = 1, 2, \ldots, n, \ldots\), \(j = 1, 2, \ldots, m, \ldots\)

\pause

Podemos pensar que um \textbf{vetor aleatório bidimensional discreto} é
um vetor formado por \textbf{duas variáveis aleatórias discretas}
definidas no \textbf{mesmo espaço amostral}.

\pause

De forma análoga, podemos definir um \textbf{vetor aleatório
\(n\)−dimensional discreto} como sendo um vetor formado por
\textbf{\(n\) variáveis aleatórias discretas} definidas no \textbf{mesmo
espaço amostral}.
\end{frame}

\begin{frame}{Função de Probabilidade Conjunta}
\phantomsection\label{funuxe7uxe3o-de-probabilidade-conjunta}
\textbf{Definição:} Seja \((X, Y)\) uma \textbf{variável aleatória
discreta bidimensional}. A cada resultado possível \((x_i, y_j)\)
associaremos um número \(p(x_i, y_j)\) representando\\
\(P(X = x_i,\; Y = y_j)\) e satisfazendo as seguintes condições:

\begin{enumerate}
\item
  \(p(x_i, y_j) \geq 0\), para todo \((x_i, y_j)\);
\item
  \(\displaystyle \sum_{j=1}^{\infty} \sum_{i=1}^{\infty} p(x_i, y_j) = 1\)
\end{enumerate}

\pause

A função \(p\) definida para todo \((x_i, y_j)\) no contradomínio de
\((X, Y)\) é denominada \textbf{função de probabilidade conjunta} de
\((X, Y)\).

\pause

O conjunto dos termos
\(\{(x_i, y_j,\; p(x_i, y_j)),\; i = 1, 2, \ldots,\; j = 1, 2, \ldots\}\)
é denominado \textbf{distribuição de probabilidade conjunta} de
\((X, Y)\).
\end{frame}

\begin{frame}{Função de Probabilidade Conjunta}
\phantomsection\label{funuxe7uxe3o-de-probabilidade-conjunta-1}
Para vetores \(n\)-dimensionais, a função de probabilidade conjunta é
\(P(X_1 = x_1,\; X_2 = x_2,\; \ldots,\; X_n = x_n)\) e satisfaz

\[
\sum_{x_1} \sum_{x_2} \cdots \sum_{x_n}
P(X_1 = x_1,\; X_2 = x_2,\; \ldots,\; X_n = x_n) = 1
\]
\end{frame}

\begin{frame}{Função de Probabilidade Conjunta}
\phantomsection\label{funuxe7uxe3o-de-probabilidade-conjunta-2}
\textbf{Exemplo:} Considere o experimento aleatório que consiste em
sortear duas bolas, sem reposição, de uma urna que contém 3 bolas
vermelhas (V) e 2 bolas brancas (B). Defina as seguintes variáveis
aleatórias:

\begin{itemize}
\item
  \(X\): o número de bolas brancas observadas.
\item
  \(Y\): a cor da segunda bola sorteada, em que \(1\) se a bola for V e
  \(0\), se for B.
\end{itemize}

\pause

Temos que, o espaço amostral do experimento

\[
S = \{(1B, 2B), (1B, 2V), (1V, 2B), (1V, 2V)\}
\]

\pause

Note que,

\[X=\{0,1,2\}, \quad \text{e} \quad Y=\{0,1\}\]
\end{frame}

\begin{frame}{Função de Probabilidade Conjunta}
\phantomsection\label{funuxe7uxe3o-de-probabilidade-conjunta-3}
Além disso,

\[
P(1B, 2B) \;=\; P(X = 2, Y = 0) = P(1B)\,P(2B \mid 1B)
= \frac{2}{5} \times \frac{1}{4}
= \frac{1}{10}
\]

\[
P(1B, 2V) \;=\; P(X = 1, Y = 1) = P(1B)\,P(2V \mid 1B)
= \frac{2}{5} \times \frac{3}{4}
= \frac{3}{10}
\]

\[
P(1V, 2B) \;=\; P(X = 1, Y = 0) = P(1V)\,P(2B \mid 1V)
= \frac{3}{5} \times \frac{2}{4}
= \frac{3}{10}
\]

\[
P(1V, 2V) \;=\; P(X = 0, Y = 1) = P(1V)\,P(2V \mid 1V)
= \frac{3}{5} \times \frac{2}{4}
= \frac{3}{10}
\]
\end{frame}

\begin{frame}{Função de Probabilidade Conjunta}
\phantomsection\label{funuxe7uxe3o-de-probabilidade-conjunta-4}
\[
\begin{array}{c|cc|c}
\hline
X \backslash Y & 0\;(\text{B}) & 1\;(\text{V}) & P(X=x) \\
\hline
0 & 0 & \dfrac{3}{10} & \dfrac{3}{10} \\
1 & \dfrac{1}{10} & \dfrac{3}{10} & \dfrac{4}{10} \\
2 & \dfrac{3}{10} & 0 & \dfrac{3}{10} \\
\hline
P(Y=y) & \dfrac{4}{10} & \dfrac{6}{10} & 1 \\
\hline
\end{array}
\]
\end{frame}




\end{document}

% Options for packages loaded elsewhere
% Options for packages loaded elsewhere
\PassOptionsToPackage{unicode}{hyperref}
\PassOptionsToPackage{hyphens}{url}
%
\documentclass[
  ignorenonframetext,
  aspectratio=169,
]{beamer}
\newif\ifbibliography
\usepackage{pgfpages}
\setbeamertemplate{caption}[numbered]
\setbeamertemplate{caption label separator}{: }
\setbeamercolor{caption name}{fg=normal text.fg}
\beamertemplatenavigationsymbolsempty
% remove section numbering
\setbeamertemplate{part page}{
  \centering
  \begin{beamercolorbox}[sep=16pt,center]{part title}
    \usebeamerfont{part title}\insertpart\par
  \end{beamercolorbox}
}
\setbeamertemplate{section page}{
  \centering
  \begin{beamercolorbox}[sep=12pt,center]{section title}
    \usebeamerfont{section title}\insertsection\par
  \end{beamercolorbox}
}
\setbeamertemplate{subsection page}{
  \centering
  \begin{beamercolorbox}[sep=8pt,center]{subsection title}
    \usebeamerfont{subsection title}\insertsubsection\par
  \end{beamercolorbox}
}
% Prevent slide breaks in the middle of a paragraph
\widowpenalties 1 10000
\raggedbottom
\AtBeginPart{
  \frame{\partpage}
}
\AtBeginSection{
  \ifbibliography
  \else
    \frame{\sectionpage}
  \fi
}
\AtBeginSubsection{
  \frame{\subsectionpage}
}
\usepackage{iftex}
\ifPDFTeX
  \usepackage[T1]{fontenc}
  \usepackage[utf8]{inputenc}
  \usepackage{textcomp} % provide euro and other symbols
\else % if luatex or xetex
  \usepackage{unicode-math} % this also loads fontspec
  \defaultfontfeatures{Scale=MatchLowercase}
  \defaultfontfeatures[\rmfamily]{Ligatures=TeX,Scale=1}
\fi
\usepackage{lmodern}

\ifPDFTeX\else
  % xetex/luatex font selection
\fi
% Use upquote if available, for straight quotes in verbatim environments
\IfFileExists{upquote.sty}{\usepackage{upquote}}{}
\IfFileExists{microtype.sty}{% use microtype if available
  \usepackage[]{microtype}
  \UseMicrotypeSet[protrusion]{basicmath} % disable protrusion for tt fonts
}{}
\makeatletter
\@ifundefined{KOMAClassName}{% if non-KOMA class
  \IfFileExists{parskip.sty}{%
    \usepackage{parskip}
  }{% else
    \setlength{\parindent}{0pt}
    \setlength{\parskip}{6pt plus 2pt minus 1pt}}
}{% if KOMA class
  \KOMAoptions{parskip=half}}
\makeatother


\usepackage{longtable,booktabs,array}
\usepackage{calc} % for calculating minipage widths
\usepackage{caption}
% Make caption package work with longtable
\makeatletter
\def\fnum@table{\tablename~\thetable}
\makeatother
\usepackage{graphicx}
\makeatletter
\newsavebox\pandoc@box
\newcommand*\pandocbounded[1]{% scales image to fit in text height/width
  \sbox\pandoc@box{#1}%
  \Gscale@div\@tempa{\textheight}{\dimexpr\ht\pandoc@box+\dp\pandoc@box\relax}%
  \Gscale@div\@tempb{\linewidth}{\wd\pandoc@box}%
  \ifdim\@tempb\p@<\@tempa\p@\let\@tempa\@tempb\fi% select the smaller of both
  \ifdim\@tempa\p@<\p@\scalebox{\@tempa}{\usebox\pandoc@box}%
  \else\usebox{\pandoc@box}%
  \fi%
}
% Set default figure placement to htbp
\def\fps@figure{htbp}
\makeatother





\setlength{\emergencystretch}{3em} % prevent overfull lines

\providecommand{\tightlist}{%
  \setlength{\itemsep}{0pt}\setlength{\parskip}{0pt}}



 


\makeatletter
\@ifpackageloaded{tcolorbox}{}{\usepackage[skins,breakable]{tcolorbox}}
\@ifpackageloaded{fontawesome5}{}{\usepackage{fontawesome5}}
\definecolor{quarto-callout-color}{HTML}{909090}
\definecolor{quarto-callout-note-color}{HTML}{0758E5}
\definecolor{quarto-callout-important-color}{HTML}{CC1914}
\definecolor{quarto-callout-warning-color}{HTML}{EB9113}
\definecolor{quarto-callout-tip-color}{HTML}{00A047}
\definecolor{quarto-callout-caution-color}{HTML}{FC5300}
\definecolor{quarto-callout-color-frame}{HTML}{acacac}
\definecolor{quarto-callout-note-color-frame}{HTML}{4582ec}
\definecolor{quarto-callout-important-color-frame}{HTML}{d9534f}
\definecolor{quarto-callout-warning-color-frame}{HTML}{f0ad4e}
\definecolor{quarto-callout-tip-color-frame}{HTML}{02b875}
\definecolor{quarto-callout-caution-color-frame}{HTML}{fd7e14}
\makeatother
\makeatletter
\@ifpackageloaded{caption}{}{\usepackage{caption}}
\AtBeginDocument{%
\ifdefined\contentsname
  \renewcommand*\contentsname{Table of contents}
\else
  \newcommand\contentsname{Table of contents}
\fi
\ifdefined\listfigurename
  \renewcommand*\listfigurename{List of Figures}
\else
  \newcommand\listfigurename{List of Figures}
\fi
\ifdefined\listtablename
  \renewcommand*\listtablename{List of Tables}
\else
  \newcommand\listtablename{List of Tables}
\fi
\ifdefined\figurename
  \renewcommand*\figurename{Figure}
\else
  \newcommand\figurename{Figure}
\fi
\ifdefined\tablename
  \renewcommand*\tablename{Table}
\else
  \newcommand\tablename{Table}
\fi
}
\@ifpackageloaded{float}{}{\usepackage{float}}
\floatstyle{ruled}
\@ifundefined{c@chapter}{\newfloat{codelisting}{h}{lop}}{\newfloat{codelisting}{h}{lop}[chapter]}
\floatname{codelisting}{Listing}
\newcommand*\listoflistings{\listof{codelisting}{List of Listings}}
\makeatother
\makeatletter
\makeatother
\makeatletter
\@ifpackageloaded{caption}{}{\usepackage{caption}}
\@ifpackageloaded{subcaption}{}{\usepackage{subcaption}}
\makeatother
\newcounter{quartocallouttipno}
\newcommand{\quartocallouttip}[1]{\refstepcounter{quartocallouttipno}\label{#1}}

\usepackage{bookmark}
\IfFileExists{xurl.sty}{\usepackage{xurl}}{} % add URL line breaks if available
\urlstyle{same}
\hypersetup{
  pdftitle={Função Geradora de Momentos},
  hidelinks,
  pdfcreator={LaTeX via pandoc}}



\title{Função Geradora de Momentos}
\author{}
\date{}

\begin{document}
\frame{\titlepage}


\begin{frame}{Momentos}
\phantomsection\label{momentos}
Calculamos algumas características de uma variável aleatória \(X\), tais
como \(E(X)\) e \(Var(X)\), através da distribuição de probabilidade de
\(X\)

\pause

Vimos que a variância pode ser expressa como uma \textbf{função da
esperança} das duas primeiras potências de \(X\), ou seja,

\[Var(X) = E(X^2) - \Big[E(X)\Big]^2\]

\pause

Outras características da distribuição de probabilidade de \(X\) podem
ser expressas por meio das esperanças das potências de \(X\) como por
exemplo \textbf{coeficientes de assimetria} e \textbf{curtose}.
\end{frame}

\begin{frame}{Momentos}
\phantomsection\label{momentos-1}
\begin{tcolorbox}[enhanced jigsaw, opacityback=0, opacitybacktitle=0.6, toptitle=1mm, breakable, rightrule=.15mm, coltitle=black, colback=white, title=\textcolor{quarto-callout-note-color}{\faInfo}\hspace{0.5em}{Definição 01: Momentos}, colframe=quarto-callout-note-color-frame, bottomtitle=1mm, leftrule=.75mm, titlerule=0mm, arc=.35mm, colbacktitle=quarto-callout-note-color!10!white, bottomrule=.15mm, toprule=.15mm, left=2mm]

Seja \(X\) uma variável aleatória. Então, o \textbf{k-ésimo momento} de
\(X\), denotado por \(\mu'_k\) é definido como,

\[\mu'_k = E(X^k)\]

desde que essa quantidade exista. O \textbf{k-ésimo momento central} de
uma variável aleatória \(X\), denotado por \(\mu_k\) é definido como,

\[\mu^k = E\Big[X - E(X)\Big]^k\]

sempre que essa quantidade existir.

\end{tcolorbox}
\end{frame}

\begin{frame}{Momentos}
\phantomsection\label{momentos-2}
Note que,

\begin{itemize}
\tightlist
\item
  \(E(X) = \mu'_1\)
\end{itemize}

\pause

\begin{itemize}
\tightlist
\item
  \(Var(X) = \mu_2 = \mu'_2 - [\mu'_1]^2\)
\end{itemize}

\pause

\begin{itemize}
\tightlist
\item
  Para qualquer variável aleatória, \(\mu_1 = 0\).
\end{itemize}
\end{frame}

\begin{frame}{Momentos}
\phantomsection\label{momentos-3}
\begin{itemize}
\tightlist
\item
  Se \(X\) é uma variável aleatória discreta,
\end{itemize}

\[\mu'_k = E(X^k) = \sum_{i=1}^{\infty} x_i^k p(x_i)\]

\pause

\begin{itemize}
\tightlist
\item
  Se \(X\) é uma variável aleatória contínua,
\end{itemize}

\[\mu'_k = E(X^k) = \int_{-\infty}^{\infty} x_i^k f(x) \, dx\]
\end{frame}

\begin{frame}{Momentos}
\phantomsection\label{momentos-4}
\begin{tcolorbox}[enhanced jigsaw, opacityback=0, opacitybacktitle=0.6, toptitle=1mm, breakable, rightrule=.15mm, coltitle=black, colback=white, title=\textcolor{quarto-callout-note-color}{\faInfo}\hspace{0.5em}{Exemplo 01: Momentos da distribuição Gamma}, colframe=quarto-callout-note-color-frame, bottomtitle=1mm, leftrule=.75mm, titlerule=0mm, arc=.35mm, colbacktitle=quarto-callout-note-color!10!white, bottomrule=.15mm, toprule=.15mm, left=2mm]

Encontre o \(k\)-ésimo momento de \(X \sim Gamma(\alpha, \lambda)\).

\end{tcolorbox}

\pause

\textbf{Solução:} Temos que, se \(X \sim Gamma(\alpha, \lambda)\), então
sua função densidade é dada por,

\[
f(x \mid \alpha,\lambda) =
\begin{cases}
\dfrac{\lambda e^{-\lambda x}(\lambda x)^{\alpha-1}}{\Gamma(\alpha)}, & x\ge 0,\\[6pt]
0, & x<0
\end{cases}
\]
\end{frame}

\begin{frame}{Momentos}
\phantomsection\label{momentos-5}
Assim,

\[
\begin{aligned}
E(X^k) &= \int_0^\infty x^k f(x\mid \alpha,\lambda)\, dx = \int_0^\infty x^k 
\frac{\lambda e^{-\lambda x}(\lambda x)^{\alpha-1}}
{\Gamma(\alpha)}\, dx \\[6pt]
 &= \int_0^\infty x^k 
\frac{\lambda e^{-\lambda x}\lambda^{\alpha-1} \, x^{\alpha-1}}
{\Gamma(\alpha)}\, dx = \frac{\lambda^\alpha}{\Gamma(\alpha)}
\int_0^\infty x^{\alpha+k-1} e^{-\lambda x}\, dx
\end{aligned}
\]

Note que a integral é quase a função gama, a não ser pelo termo
\(e^{-\lambda x}\). Seja então a seguinte mudança de variável,

\[u = \lambda  x \,\,\, \Rightarrow \,\,\, x = \dfrac{u}{\lambda}, \qquad dx = \dfrac{du}{\lambda}\]
\end{frame}

\begin{frame}{Momentos}
\phantomsection\label{momentos-6}
Assim,

\[
\begin{aligned}
E(X^k) &=  \dfrac{\lambda^\alpha}{\Gamma(\alpha)}
\int_0^\infty x^{\alpha+k-1} e^{-\lambda x}\, dx  = \dfrac{\lambda^\alpha}{\Gamma(\alpha)}
\int_0^\infty \Big(\dfrac{u}{\lambda}\Big)^{\alpha+k-1} e^{-u}\, \dfrac{du}{\lambda} \\[6pt] &= \dfrac{\lambda^\alpha}{\lambda ^{\alpha + k}\,\Gamma(\alpha)}
\int_0^\infty u^{\alpha+k-1} e^{-u}\, du = \dfrac{\lambda^\alpha \Gamma(\alpha + k)}{\lambda ^{\alpha + k}\,\Gamma(\alpha)} \\[6pt] &= \dfrac{\Gamma(\alpha+k)}{\Gamma(\alpha)\lambda^k}
\end{aligned}
\]

Mas,

\[
\Gamma(\alpha+k) = (\alpha+k-1)(\alpha+k-2)\cdots(\alpha+1)\alpha\Gamma(\alpha)
\]
\end{frame}

\begin{frame}{Momentos}
\phantomsection\label{momentos-7}
De forma que, o \(k\)-ésimo momento de uma variável aleatória
\(X \sim Gamma (\alpha, \lambda)\) é dado por

\[
E(X^k)
=
\frac{\alpha(\alpha+1)\cdots(\alpha+k-1)}{\lambda^k}
\]
\end{frame}

\begin{frame}{Momentos}
\phantomsection\label{momentos-8}
\begin{tcolorbox}[enhanced jigsaw, opacityback=0, opacitybacktitle=0.6, toptitle=1mm, breakable, rightrule=.15mm, coltitle=black, colback=white, title=\textcolor{quarto-callout-note-color}{\faInfo}\hspace{0.5em}{Exemplo 02: Momentos da distribuição Weibull}, colframe=quarto-callout-note-color-frame, bottomtitle=1mm, leftrule=.75mm, titlerule=0mm, arc=.35mm, colbacktitle=quarto-callout-note-color!10!white, bottomrule=.15mm, toprule=.15mm, left=2mm]

Encontre o \(k\)-ésimo momento de \(X \sim Weibull(\alpha, \beta)\).

\end{tcolorbox}

\pause

\textbf{Solução:} Temos que, se \(X \sim Weibull(\alpha, \beta)\), então
sua função densidade é dada por,

\[
f(x \mid \alpha,\beta)=
\begin{cases}
\dfrac{\beta}{\alpha}
\left(\dfrac{x}{\alpha}\right)^{\beta-1}
\exp\!\left[-\left(\dfrac{x}{\alpha}\right)^{\beta}\right], & x\ge 0,\\[6pt]
0, & x<0.
\end{cases}
\]
\end{frame}

\begin{frame}{Momentos}
\phantomsection\label{momentos-9}
Assim,

\[
\begin{aligned}
E(X^k) &= \int_0^\infty x^k 
\frac{\beta}{\alpha}
\left(\frac{x}{\alpha}\right)^{\beta-1}
\exp\!\left[-\left(\frac{x}{\alpha}\right)^{\beta}\right] dx 
\end{aligned}
\]

Mudança de variável:

\[
u = \left(\frac{x}{\alpha}\right)^{\beta}
\quad\Rightarrow\quad
x = \alpha u^{1/\beta},\qquad
dx = \alpha \frac{1}{\beta} u^{1/\beta - 1} du
\]

Além disso:

\[
x^k = \alpha^k u^{k/\beta},
\qquad
\left(\frac{x}{\alpha}\right)^{\beta-1}=u^{(\beta-1)/\beta},
\qquad
\exp\!\left[-\left(\frac{x}{\alpha}\right)^{\beta}\right]=e^{-u}
\]
\end{frame}

\begin{frame}{Momentos}
\phantomsection\label{momentos-10}
logo,

\[
\begin{aligned}
E(X^k) &= \int_0^\infty x^k 
\frac{\beta}{\alpha}
\left(\frac{x}{\alpha}\right)^{\beta-1}
\exp\!\left[-\left(\frac{x}{\alpha}\right)^{\beta}\right] dx \\[6pt] &= \int_0^\infty 
\alpha^k u^{k/\beta}\,
\frac{\beta}{\alpha}\,
u^{(\beta-1)/\beta}\,
e^{-u}\,
\alpha \frac{1}{\beta}u^{1/\beta -1}\,du \\[6pt] &= \alpha^k \int_0^\infty 
u^{\frac{k+\beta}{\beta}-1} 
e^{-u}\,du = \alpha^k\,\,\Gamma\!\left(1+\frac{k}{\beta}\right)
\end{aligned}
\]
\end{frame}

\begin{frame}{Função Geradora de Momentos}
\phantomsection\label{funuxe7uxe3o-geradora-de-momentos}
\begin{tcolorbox}[enhanced jigsaw, opacityback=0, opacitybacktitle=0.6, toptitle=1mm, breakable, rightrule=.15mm, coltitle=black, colback=white, title=\textcolor{quarto-callout-note-color}{\faInfo}\hspace{0.5em}{Definição 02: Função Geradora de Momentos}, colframe=quarto-callout-note-color-frame, bottomtitle=1mm, leftrule=.75mm, titlerule=0mm, arc=.35mm, colbacktitle=quarto-callout-note-color!10!white, bottomrule=.15mm, toprule=.15mm, left=2mm]

Seja \(X\) uma variável aleatória qualquer. A \textbf{função geradora de
momentos} (FGM) de \(X\), denotada por \(M_X\), é definida por

\[
M_X(t)=E(e^{tX}),
\]

para valores de \(t\) em um intervalo contendo \(0\) onde a experença
exista.

\end{tcolorbox}

\textbf{Importante:} a função geradora de momentos é função de \(t\).
Para ela existir basta que exista \(\varepsilon > 0\) tal que
\(E(e^{tX})\) esteja bem definida para qualquer
\(t \in (-\varepsilon, \varepsilon)\).
\end{frame}

\begin{frame}{Função Geradora de Momentos}
\phantomsection\label{funuxe7uxe3o-geradora-de-momentos-1}
Note que a \textbf{definição de função geradora de momentos} é feita
independente do tipo de variável, mas a forma de encontrá-la depende se
a variável for \textbf{discreta} ou \textbf{contínua}, isto é,

\begin{itemize}
\tightlist
\item
  Se \(X\) é discreta,
\end{itemize}

\[M_X(t) = E(e^{tX}) = \sum_{\forall x \in S_X} \, e^{tx} p_X(x)\]

\begin{itemize}
\tightlist
\item
  Se \(X\) é contínua,
\end{itemize}

\[M_X(t) = E(e^{tX}) = \int_{-\infty}^{\infty} \, e^{tx} f_X(x) \, dx\]
\end{frame}

\begin{frame}{Função Geradora de Momentos}
\phantomsection\label{funuxe7uxe3o-geradora-de-momentos-2}
\begin{tcolorbox}[enhanced jigsaw, opacityback=0, opacitybacktitle=0.6, toptitle=1mm, breakable, rightrule=.15mm, coltitle=black, colback=white, title=\textcolor{quarto-callout-note-color}{\faInfo}\hspace{0.5em}{Teorema 01}, colframe=quarto-callout-note-color-frame, bottomtitle=1mm, leftrule=.75mm, titlerule=0mm, arc=.35mm, colbacktitle=quarto-callout-note-color!10!white, bottomrule=.15mm, toprule=.15mm, left=2mm]

Suponha que a função geradora de momentos de \(X\) exista para
\(|t| < \varepsilon\), \(\varepsilon > 0\). Então, \(E(X^k)\) existe
para \(k = 1, 2, \cdots\) e temos:

\[E(X^k) = \dfrac{d^k}{dt^k} M_X(t) \Bigg|_{t=0}\]

ou seja, o \(k\)-ésimo momento de \(X\) é igual à derivada de ordem
\(k\) de \(M_X(t)\) avaliada em \(t = 0\).

\end{tcolorbox}

\textbf{Demonstração:} Suponha que a função geradora de momentos de
\(X\) exista para todo \(t\) tal que \(|t|<\varepsilon\), com
\(\varepsilon>0\), isto é,

\[
M_X(t)=E(e^{tX})<\infty,\qquad |t|<\varepsilon.
\]
\end{frame}

\begin{frame}{Função Geradora de Momentos}
\phantomsection\label{funuxe7uxe3o-geradora-de-momentos-3}
Pela série de Maclaurin da função exponencial, para qualquer número real
\(y\) temos \[
e^{y} = \sum_{n=0}^{\infty} \frac{y^n}{n!}
\]

Aplicando isso a \(y=tX\), obtemos, para cada \(t\) com
\(|t|<\varepsilon\), \[
e^{tX} = \sum_{n=0}^{\infty} \frac{(tX)^n}{n!} = 1 + tX +\frac{(tX)^2}{2!} + \frac{(tX)^3}{3!} + \cdots
\]

Temos que \(M_X(t)=E(e^{tX})\). Logo, admitindo ser válido permutar soma
infinita e esperança, temos
\end{frame}

\begin{frame}{Função Geradora de Momentos}
\phantomsection\label{funuxe7uxe3o-geradora-de-momentos-4}
\[
M_X(t)
= E(e^{tX})
= E\left(\sum_{n=0}^{\infty} \frac{(tX)^n}{n!}\right)
= \sum_{n=0}^{\infty} \frac{t^n}{n!} E(X^n),
\qquad |t|<\varepsilon
\]

Portanto, \(M_X(t)\) é dada, em uma vizinhança de \(0\), por uma série
de potência da forma \[
M_X(t) = \sum_{n=0}^{\infty} a_n t^n,
\qquad \text{com } a_n = \frac{E(X^n)}{n!}
\]
\end{frame}

\begin{frame}{Função Geradora de Momentos}
\phantomsection\label{funuxe7uxe3o-geradora-de-momentos-5}
Da teoria de séries de potência, sabemos que, se \[
M_X(t) = \sum_{n=0}^{\infty} a_n t^n,
\] então a \(k\)-ésima derivada é \[
M_X^{(k)}(t) = \sum_{n=k}^{\infty} a_n\, n(n-1)\cdots (n-k+1)\, t^{\,n-k}
\]
\end{frame}

\begin{frame}{Função Geradora de Momentos}
\phantomsection\label{funuxe7uxe3o-geradora-de-momentos-6}
Agora substituímos \(t=0\):

Observe:

\begin{itemize}
\tightlist
\item
  Se \(n > k\), aparece o fator \(t^{n-k} = 0^{n-k} = 0\);\\
\item
  Então \textbf{todos} os termos com \(n>k\) desaparecem;
\item
  Só o termo com \(n=k\) permanece.
\end{itemize}

O único termo sobrevivente é:

\[
a_k \, k(k-1)(k-2)\cdots 1 \, t^{\,0}
= a_k \, k!
\]
\end{frame}

\begin{frame}{Função Geradora de Momentos}
\phantomsection\label{funuxe7uxe3o-geradora-de-momentos-7}
Assim,

\[
M_X^{(k)}(0) = a_k\, k!
= \frac{E(X^k)}{k!}\,k!
= E(X^k)
\]

Logo, \[
E(X^k) = \left.\frac{d^k}{dt^k} M_X(t)\right|_{t=0},
\qquad k=1,2,\dots
\]

o que mostra que o \(k\)-ésimo momento de \(X\) é igual à derivada de
ordem \(k\) da função geradora de momentos avaliada em \(t=0\).
\end{frame}

\begin{frame}{Função Geradora de Momentos}
\phantomsection\label{funuxe7uxe3o-geradora-de-momentos-8}
\begin{tcolorbox}[enhanced jigsaw, opacityback=0, opacitybacktitle=0.6, toptitle=1mm, breakable, rightrule=.15mm, coltitle=black, colback=white, title=\textcolor{quarto-callout-tip-color}{\faLightbulb}\hspace{0.5em}{Tip \ref*{tip-FGM}: Dica Importante!}, colframe=quarto-callout-tip-color-frame, bottomtitle=1mm, leftrule=.75mm, titlerule=0mm, arc=.35mm, colbacktitle=quarto-callout-tip-color!10!white, bottomrule=.15mm, toprule=.15mm, left=2mm]

\quartocallouttip{tip-FGM} 

Para qualquer variável aleatória \(X\):

\[
M_X(0) = E(e^{0X}) = 1.
\]

Isso sempre deve ocorrer. Use esse fato para verificar se sua FGM está
correta.

\end{tcolorbox}
\end{frame}

\begin{frame}{Função Geradora de Momentos}
\phantomsection\label{funuxe7uxe3o-geradora-de-momentos-9}
\begin{tcolorbox}[enhanced jigsaw, opacityback=0, opacitybacktitle=0.6, toptitle=1mm, breakable, rightrule=.15mm, coltitle=black, colback=white, title=\textcolor{quarto-callout-note-color}{\faInfo}\hspace{0.5em}{Exemplo 03: Distribuição de Bernoulli}, colframe=quarto-callout-note-color-frame, bottomtitle=1mm, leftrule=.75mm, titlerule=0mm, arc=.35mm, colbacktitle=quarto-callout-note-color!10!white, bottomrule=.15mm, toprule=.15mm, left=2mm]

Seja \(X \sim Bernoulli(p)\). Encontre sua função geradora de momentos e
a partir dela, encontre \(E(X)\) e \(\operatorname{Var}(X)\).

\end{tcolorbox}

\pause

\textbf{Solução:} Por definição,

\[
M_X(t) = E(e^{tX}) = \sum_{x=0}^1 e^{tx} p^x(1-p)^{1-x}
\]

Como \(X\) só assume os valores \(0\) e \(1\):

\[
M_X(t)
= e^{t\cdot 0} p^0(1-p)^{1-0} + e^{t\cdot 1} p^1(1-p)^{1-1}
= (1-p) + e^{t}p
= 1 - p +  e^{t} p
\]
\end{frame}

\begin{frame}{Função Geradora de Momentos}
\phantomsection\label{funuxe7uxe3o-geradora-de-momentos-10}
Portanto,

\[
\boxed{M_X(t) = 1 - p + p e^{t}, \quad t \in \mathbb{R}}
\]

Veja que pela Tip~\ref{tip-FGM}
\(M_X(0) = 1 - p + p e^{0} = 1 - p + p \times 1 = 1\). Assim,

\begin{itemize}
\tightlist
\item
  Primeira derivada de \(M_X(t)\):
\end{itemize}

\[
M_X'(t) = \frac{d}{dt}\big(1 - p + p e^{t}\big)
= p e^{t}
\]

Avaliada em \(t = 0\), temos \(E(X) = M_X'(0) = p e^{0} = p\)
\end{frame}

\begin{frame}{Função Geradora de Momentos}
\phantomsection\label{funuxe7uxe3o-geradora-de-momentos-11}
\begin{itemize}
\tightlist
\item
  Segunda derivada de \(M_X(t)\):
\end{itemize}

\[
M_X''(t) = \frac{d}{dt}\big(M_X'(t)\big)
= \frac{d}{dt}(p e^{t}) = p e^{t}
\]

Avaliada em \(t = 0\), temos \(E(X) = M_X''(0) = p e^{0} = p\).

Assim,

\[
\operatorname{Var}(X)
= M_X''(0) - [M_X'(0)]^2
= p - p^2
= p(1-p)
\]
\end{frame}

\begin{frame}{Função Geradora de Momentos}
\phantomsection\label{funuxe7uxe3o-geradora-de-momentos-12}
\begin{tcolorbox}[enhanced jigsaw, opacityback=0, opacitybacktitle=0.6, toptitle=1mm, breakable, rightrule=.15mm, coltitle=black, colback=white, title=\textcolor{quarto-callout-note-color}{\faInfo}\hspace{0.5em}{Exemplo 03: Distribuição Binomial}, colframe=quarto-callout-note-color-frame, bottomtitle=1mm, leftrule=.75mm, titlerule=0mm, arc=.35mm, colbacktitle=quarto-callout-note-color!10!white, bottomrule=.15mm, toprule=.15mm, left=2mm]

Seja \(X \sim Binomial(n,p)\). Encontre sua função geradora de momentos
e a partir dela, encontre \(E(X)\) e \(\operatorname{Var}(X)\).

\end{tcolorbox}

\pause

\textbf{Solução:} Se \(X \sim Binomial(n,p)\) então sua f.p. é dada por

\[
P(X = k) = \binom{n}{k} p^k (1-p)^{n-k}, \qquad k=0,1,\dots,n
\]

Por definição,

\[
M_X(t) = E(e^{tX}) = \sum_{k=0}^{n} e^{tk} P(X = k)
= \sum_{k=0}^{n} e^{tk} \binom{n}{k} p^k (1-p)^{n-k}
\]
\end{frame}

\begin{frame}{Função Geradora de Momentos}
\phantomsection\label{funuxe7uxe3o-geradora-de-momentos-13}
Escrevendo \(e^{tk} = (e^{t})^k\):

\[
M_X(t)
= \sum_{k=0}^{n} \binom{n}{k} (p e^t)^k (1-p)^{n-k}
\]

Temos que o binômio de Newton é dado por:

\[
(a+b)^n = \sum_{k=0}^{n} \binom{n}{k} a^k b^{n-k}
\]
\end{frame}

\begin{frame}{Função Geradora de Momentos}
\phantomsection\label{funuxe7uxe3o-geradora-de-momentos-14}
Assim, tomando \(a = p e^{t}\) e \(b = 1-p\):

\[
M_X(t) = (1-p + p e^{t})^n
\]

Portanto,

\[
\boxed{M_X(t) = (1-p + p e^{t})^n,\quad t\in\mathbb{R}}
\]

Veja que pela Tip~\ref{tip-FGM}
\(M_X(0) = (1-p + p e^{0})^n = (1 - p + p \times 1)^n = 1^n = 1\).
\end{frame}

\begin{frame}{Função Geradora de Momentos}
\phantomsection\label{funuxe7uxe3o-geradora-de-momentos-15}
Sabemos que

\[
E(X) = M_X'(0)
\]

Derivada primeira em relação a \(t\),

\[
M_X'(t) = \frac{d}{dt} (1-p + p e^{t})^n
= n(1-p + p e^{t})^{n-1} \cdot p e^{t}
\]

Logo,

\[
M_X'(0)
= n(1-p + p e^{0})^{n-1} \cdot p e^{0}
= n(1-p + p)^{n-1} p
= n p
\]
\end{frame}

\begin{frame}{Função Geradora de Momentos}
\phantomsection\label{funuxe7uxe3o-geradora-de-momentos-16}
Vamos agora encontrar a segunda derivada. Temos

\[
M_X'(t) = n p e^{t} (1-p + p e^{t})^{n-1}
\]

Aplicando a regra do produto:

\[
\begin{aligned}
M_X''(t)
&= \frac{d}{dt}\Big[ n p e^{t} (1-p + p e^{t})^{n-1} \Big] \\
&= n p e^{t} (n-1)(1-p + p e^{t})^{n-2} \cdot p e^{t}
 \;+\; n p e^{t} (1-p + p e^{t})^{n-1}
\end{aligned}
\]
\end{frame}

\begin{frame}{Função Geradora de Momentos}
\phantomsection\label{funuxe7uxe3o-geradora-de-momentos-17}
Avaliando em \(t=0\):

\begin{itemize}
\tightlist
\item
  \(e^{0}=1\)\\
\item
  \(1-p + p e^{0} = 1-p + p = 1\)
\end{itemize}

Assim,

\[
\begin{aligned}
M_X''(0)
&= n p \cdot 1 \cdot (n-1) \cdot 1^{\,n-2} \cdot p \cdot 1
   \;+\; n p \cdot 1 \cdot 1^{\,n-1} \\[6pt]
&= n p (n-1)p + n p \\[6pt]
&= n p \big[(n-1)p + 1\big]
\end{aligned}
\]
\end{frame}

\begin{frame}{Função Geradora de Momentos}
\phantomsection\label{funuxe7uxe3o-geradora-de-momentos-18}
De forma que,

\[
\begin{aligned}
\operatorname{Var}(X)
&= M_X''(0) - [M_X'(0)]^2\\[6pt] 
&= n p \big[(n-1)p + 1\big] - (np)^2 \\[6pt]
&= n p \big[(n-1)p + 1 - n p\big] \\[6pt]
&= n p (1 - p)
\end{aligned}
\]
\end{frame}

\begin{frame}{Função Geradora de Momentos}
\phantomsection\label{funuxe7uxe3o-geradora-de-momentos-19}
\begin{tcolorbox}[enhanced jigsaw, opacityback=0, opacitybacktitle=0.6, toptitle=1mm, breakable, rightrule=.15mm, coltitle=black, colback=white, title=\textcolor{quarto-callout-note-color}{\faInfo}\hspace{0.5em}{Exemplo 04: Distribuição Geométrica}, colframe=quarto-callout-note-color-frame, bottomtitle=1mm, leftrule=.75mm, titlerule=0mm, arc=.35mm, colbacktitle=quarto-callout-note-color!10!white, bottomrule=.15mm, toprule=.15mm, left=2mm]

Seja \(X \sim Geo(p)\). Encontre sua função geradora de momentos e a
partir dela, encontre \(E(X)\) e \(\operatorname{Var}(X)\).

\end{tcolorbox}

\pause

\textbf{Solução:}
\end{frame}




\end{document}

% Options for packages loaded elsewhere
% Options for packages loaded elsewhere
\PassOptionsToPackage{unicode}{hyperref}
\PassOptionsToPackage{hyphens}{url}
%
\documentclass[
  ignorenonframetext,
  aspectratio=169,
]{beamer}
\newif\ifbibliography
\usepackage{pgfpages}
\setbeamertemplate{caption}[numbered]
\setbeamertemplate{caption label separator}{: }
\setbeamercolor{caption name}{fg=normal text.fg}
\beamertemplatenavigationsymbolsempty
% remove section numbering
\setbeamertemplate{part page}{
  \centering
  \begin{beamercolorbox}[sep=16pt,center]{part title}
    \usebeamerfont{part title}\insertpart\par
  \end{beamercolorbox}
}
\setbeamertemplate{section page}{
  \centering
  \begin{beamercolorbox}[sep=12pt,center]{section title}
    \usebeamerfont{section title}\insertsection\par
  \end{beamercolorbox}
}
\setbeamertemplate{subsection page}{
  \centering
  \begin{beamercolorbox}[sep=8pt,center]{subsection title}
    \usebeamerfont{subsection title}\insertsubsection\par
  \end{beamercolorbox}
}
% Prevent slide breaks in the middle of a paragraph
\widowpenalties 1 10000
\raggedbottom
\AtBeginPart{
  \frame{\partpage}
}
\AtBeginSection{
  \ifbibliography
  \else
    \frame{\sectionpage}
  \fi
}
\AtBeginSubsection{
  \frame{\subsectionpage}
}
\usepackage{iftex}
\ifPDFTeX
  \usepackage[T1]{fontenc}
  \usepackage[utf8]{inputenc}
  \usepackage{textcomp} % provide euro and other symbols
\else % if luatex or xetex
  \usepackage{unicode-math} % this also loads fontspec
  \defaultfontfeatures{Scale=MatchLowercase}
  \defaultfontfeatures[\rmfamily]{Ligatures=TeX,Scale=1}
\fi
\usepackage{lmodern}

\ifPDFTeX\else
  % xetex/luatex font selection
\fi
% Use upquote if available, for straight quotes in verbatim environments
\IfFileExists{upquote.sty}{\usepackage{upquote}}{}
\IfFileExists{microtype.sty}{% use microtype if available
  \usepackage[]{microtype}
  \UseMicrotypeSet[protrusion]{basicmath} % disable protrusion for tt fonts
}{}
\makeatletter
\@ifundefined{KOMAClassName}{% if non-KOMA class
  \IfFileExists{parskip.sty}{%
    \usepackage{parskip}
  }{% else
    \setlength{\parindent}{0pt}
    \setlength{\parskip}{6pt plus 2pt minus 1pt}}
}{% if KOMA class
  \KOMAoptions{parskip=half}}
\makeatother


\usepackage{longtable,booktabs,array}
\usepackage{calc} % for calculating minipage widths
\usepackage{caption}
% Make caption package work with longtable
\makeatletter
\def\fnum@table{\tablename~\thetable}
\makeatother
\usepackage{graphicx}
\makeatletter
\newsavebox\pandoc@box
\newcommand*\pandocbounded[1]{% scales image to fit in text height/width
  \sbox\pandoc@box{#1}%
  \Gscale@div\@tempa{\textheight}{\dimexpr\ht\pandoc@box+\dp\pandoc@box\relax}%
  \Gscale@div\@tempb{\linewidth}{\wd\pandoc@box}%
  \ifdim\@tempb\p@<\@tempa\p@\let\@tempa\@tempb\fi% select the smaller of both
  \ifdim\@tempa\p@<\p@\scalebox{\@tempa}{\usebox\pandoc@box}%
  \else\usebox{\pandoc@box}%
  \fi%
}
% Set default figure placement to htbp
\def\fps@figure{htbp}
\makeatother





\setlength{\emergencystretch}{3em} % prevent overfull lines

\providecommand{\tightlist}{%
  \setlength{\itemsep}{0pt}\setlength{\parskip}{0pt}}



 


\makeatletter
\@ifpackageloaded{tcolorbox}{}{\usepackage[skins,breakable]{tcolorbox}}
\@ifpackageloaded{fontawesome5}{}{\usepackage{fontawesome5}}
\definecolor{quarto-callout-color}{HTML}{909090}
\definecolor{quarto-callout-note-color}{HTML}{0758E5}
\definecolor{quarto-callout-important-color}{HTML}{CC1914}
\definecolor{quarto-callout-warning-color}{HTML}{EB9113}
\definecolor{quarto-callout-tip-color}{HTML}{00A047}
\definecolor{quarto-callout-caution-color}{HTML}{FC5300}
\definecolor{quarto-callout-color-frame}{HTML}{acacac}
\definecolor{quarto-callout-note-color-frame}{HTML}{4582ec}
\definecolor{quarto-callout-important-color-frame}{HTML}{d9534f}
\definecolor{quarto-callout-warning-color-frame}{HTML}{f0ad4e}
\definecolor{quarto-callout-tip-color-frame}{HTML}{02b875}
\definecolor{quarto-callout-caution-color-frame}{HTML}{fd7e14}
\makeatother
\makeatletter
\@ifpackageloaded{caption}{}{\usepackage{caption}}
\AtBeginDocument{%
\ifdefined\contentsname
  \renewcommand*\contentsname{Table of contents}
\else
  \newcommand\contentsname{Table of contents}
\fi
\ifdefined\listfigurename
  \renewcommand*\listfigurename{List of Figures}
\else
  \newcommand\listfigurename{List of Figures}
\fi
\ifdefined\listtablename
  \renewcommand*\listtablename{List of Tables}
\else
  \newcommand\listtablename{List of Tables}
\fi
\ifdefined\figurename
  \renewcommand*\figurename{Figure}
\else
  \newcommand\figurename{Figure}
\fi
\ifdefined\tablename
  \renewcommand*\tablename{Table}
\else
  \newcommand\tablename{Table}
\fi
}
\@ifpackageloaded{float}{}{\usepackage{float}}
\floatstyle{ruled}
\@ifundefined{c@chapter}{\newfloat{codelisting}{h}{lop}}{\newfloat{codelisting}{h}{lop}[chapter]}
\floatname{codelisting}{Listing}
\newcommand*\listoflistings{\listof{codelisting}{List of Listings}}
\makeatother
\makeatletter
\makeatother
\makeatletter
\@ifpackageloaded{caption}{}{\usepackage{caption}}
\@ifpackageloaded{subcaption}{}{\usepackage{subcaption}}
\makeatother

\usepackage{bookmark}
\IfFileExists{xurl.sty}{\usepackage{xurl}}{} % add URL line breaks if available
\urlstyle{same}
\hypersetup{
  pdftitle={Função Geradora de Momentos},
  hidelinks,
  pdfcreator={LaTeX via pandoc}}



\title{Função Geradora de Momentos}
\author{}
\date{}

\begin{document}
\frame{\titlepage}


\begin{frame}{Momentos}
\phantomsection\label{momentos}
Calculamos algumas características de uma variável aleatória \(X\), tais
como \(E(X)\) e \(Var(X)\), através da distribuição de probabilidade de
\(X\)

\pause

Vimos que a variância pode ser expressa como uma \textbf{função da
esperança} das duas primeiras potências de \(X\), ou seja,

\[Var(X) = E(X^2) - \Big[E(X)\Big]^2\]

\pause

Outras características da distribuição de probabilidade de \(X\) podem
ser expressas por meio das esperanças das potências de \(X\) como por
exemplo \textbf{coeficientes de assimetria} e \textbf{curtose}.
\end{frame}

\begin{frame}{Momentos}
\phantomsection\label{momentos-1}
\begin{tcolorbox}[enhanced jigsaw, toprule=.15mm, arc=.35mm, toptitle=1mm, leftrule=.75mm, opacityback=0, left=2mm, titlerule=0mm, colframe=quarto-callout-note-color-frame, title=\textcolor{quarto-callout-note-color}{\faInfo}\hspace{0.5em}{Definição 01: Momentos}, coltitle=black, breakable, colback=white, colbacktitle=quarto-callout-note-color!10!white, bottomrule=.15mm, rightrule=.15mm, bottomtitle=1mm, opacitybacktitle=0.6]

Seja \(X\) uma variável aleatória. Então, o \textbf{k-ésimo momento} de
\(X\), denotado por \(\mu'_k\) é definido como,

\[\mu'_k = E(X^k)\]

desde que essa quantidade exista. O \textbf{k-ésimo momento central} de
uma variável aleatória \(X\), denotado por \(\mu_k\) é definido como,

\[\mu_k = E\Big[X - E(X)\Big]^k\]

sempre que essa quantidade existir.

\end{tcolorbox}
\end{frame}

\begin{frame}{Momentos}
\phantomsection\label{momentos-2}
Note que,

\begin{itemize}
\tightlist
\item
  \(E(X) = \mu'_1\)
\end{itemize}

\pause

\begin{itemize}
\tightlist
\item
  \(Var(X) = \mu_2 = \mu'_2 - [\mu'_1]^2\)
\end{itemize}

\pause

\begin{itemize}
\tightlist
\item
  Para qualquer variável aleatória, \(\mu_1 = 0\).
\end{itemize}
\end{frame}

\begin{frame}{Momentos}
\phantomsection\label{momentos-3}
\begin{itemize}
\tightlist
\item
  Se \(X\) é uma variável aleatória discreta,
\end{itemize}

\[\mu'_k = E(X^k) = \sum_{i=1}^{\infty} x_i^k p(x_i)\]

\pause

\begin{itemize}
\tightlist
\item
  Se \(X\) é uma variável aleatória contínua,
\end{itemize}

\[\mu'_k = E(X^k) = \int_{-\infty}^{\infty} x^k f(x) \, dx\]
\end{frame}

\begin{frame}{Momentos}
\phantomsection\label{momentos-4}
\begin{tcolorbox}[enhanced jigsaw, toprule=.15mm, arc=.35mm, toptitle=1mm, leftrule=.75mm, opacityback=0, left=2mm, titlerule=0mm, colframe=quarto-callout-note-color-frame, title=\textcolor{quarto-callout-note-color}{\faInfo}\hspace{0.5em}{Exemplo 01: Momentos da distribuição Gamma}, coltitle=black, breakable, colback=white, colbacktitle=quarto-callout-note-color!10!white, bottomrule=.15mm, rightrule=.15mm, bottomtitle=1mm, opacitybacktitle=0.6]

Encontre o \(k\)-ésimo momento de \(X \sim Gamma(\alpha, \lambda)\).

\end{tcolorbox}

\pause

\textbf{Solução:} Temos que, se \(X \sim Gamma(\alpha, \lambda)\), então
sua função densidade é dada por,

\[
f(x \mid \alpha,\lambda) =
\begin{cases}
\dfrac{\lambda e^{-\lambda x}(\lambda x)^{\alpha-1}}{\Gamma(\alpha)}, & x\ge 0,\\[6pt]
0, & x<0
\end{cases}
\]
\end{frame}

\begin{frame}{Momentos}
\phantomsection\label{momentos-5}
Assim,

\[
\begin{aligned}
E(X^k) &= \int_0^\infty x^k f(x\mid \alpha,\lambda)\, dx = \int_0^\infty x^k 
\frac{\lambda e^{-\lambda x}(\lambda x)^{\alpha-1}}
{\Gamma(\alpha)}\, dx \\[6pt]
 &= \int_0^\infty x^k 
\frac{\lambda e^{-\lambda x}\lambda^{\alpha-1} \, x^{\alpha-1}}
{\Gamma(\alpha)}\, dx = \frac{\lambda^\alpha}{\Gamma(\alpha)}
\int_0^\infty x^{\alpha+k-1} e^{-\lambda x}\, dx
\end{aligned}
\]

Note que a integral é quase a função gama, a não ser pelo termo
\(e^{-\lambda x}\). Seja então a seguinte mudança de variável,

\[u = \lambda  x \,\,\, \Rightarrow \,\,\, x = \dfrac{u}{\lambda}, \qquad dx = \dfrac{du}{\lambda}\]
\end{frame}

\begin{frame}{Momentos}
\phantomsection\label{momentos-6}
Assim,

\[
\begin{aligned}
E(X^k) &=  \dfrac{\lambda^\alpha}{\Gamma(\alpha)}
\int_0^\infty x^{\alpha+k-1} e^{-\lambda x}\, dx  = \dfrac{\lambda^\alpha}{\Gamma(\alpha)}
\int_0^\infty \Big(\dfrac{u}{\lambda}\Big)^{\alpha+k-1} e^{-u}\, \dfrac{du}{\lambda} \\[6pt] &= \dfrac{\lambda^\alpha}{\lambda ^{\alpha + k}\,\Gamma(\alpha)}
\int_0^\infty u^{\alpha+k-1} e^{-u}\, du = \dfrac{\lambda^\alpha \Gamma(\alpha + k)}{\lambda ^{\alpha + k}\,\Gamma(\alpha)} \\[6pt] &= \dfrac{\Gamma(\alpha+k)}{\Gamma(\alpha)\lambda^k}
\end{aligned}
\]

Mas,

\[
\Gamma(\alpha+k) = (\alpha+k-1)(\alpha+k-2)\cdots(\alpha+1)\alpha\Gamma(\alpha)
\]
\end{frame}

\begin{frame}{Momentos}
\phantomsection\label{momentos-7}
De forma que, o \(k\)-ésimo momento de uma variável aleatória
\(X \sim Gamma (\alpha, \lambda)\) é dado por

\[
E(X^k)
=
\frac{\alpha(\alpha+1)\cdots(\alpha+k-1)}{\lambda^k}
\]
\end{frame}

\begin{frame}{Momentos}
\phantomsection\label{momentos-8}
\begin{tcolorbox}[enhanced jigsaw, toprule=.15mm, arc=.35mm, toptitle=1mm, leftrule=.75mm, opacityback=0, left=2mm, titlerule=0mm, colframe=quarto-callout-note-color-frame, title=\textcolor{quarto-callout-note-color}{\faInfo}\hspace{0.5em}{Exemplo 02: Momentos da distribuição Weibull}, coltitle=black, breakable, colback=white, colbacktitle=quarto-callout-note-color!10!white, bottomrule=.15mm, rightrule=.15mm, bottomtitle=1mm, opacitybacktitle=0.6]

Encontre o \(k\)-ésimo momento de \(X \sim Weibull(\alpha, \beta)\).

\end{tcolorbox}

\pause

\textbf{Solução:} Temos que, se \(X \sim Weibull(\alpha, \beta)\), então
sua função densidade é dada por,

\[
f(x \mid \alpha,\beta)=
\begin{cases}
\dfrac{\beta}{\alpha}
\left(\dfrac{x}{\alpha}\right)^{\beta-1}
\exp\!\left[-\left(\dfrac{x}{\alpha}\right)^{\beta}\right], & x\ge 0,\\[6pt]
0, & x<0
\end{cases}
\]
\end{frame}

\begin{frame}{Momentos}
\phantomsection\label{momentos-9}
Assim,

\[
\begin{aligned}
E(X^k) &= \int_0^\infty x^k 
\frac{\beta}{\alpha}
\left(\frac{x}{\alpha}\right)^{\beta-1}
\exp\!\left[-\left(\frac{x}{\alpha}\right)^{\beta}\right] dx 
\end{aligned}
\]

Mudança de variável:

\[
u = \left(\frac{x}{\alpha}\right)^{\beta}
\quad\Rightarrow\quad
x = \alpha u^{1/\beta},\qquad
dx = \alpha \frac{1}{\beta} u^{1/\beta - 1} du
\]

Além disso:

\[
x^k = \alpha^k u^{k/\beta},
\qquad
\left(\frac{x}{\alpha}\right)^{\beta-1}=u^{(\beta-1)/\beta},
\qquad
\exp\!\left[-\left(\frac{x}{\alpha}\right)^{\beta}\right]=e^{-u}
\]
\end{frame}

\begin{frame}{Momentos}
\phantomsection\label{momentos-10}
logo,

\[
\begin{aligned}
E(X^k) &= \int_0^\infty x^k 
\frac{\beta}{\alpha}
\left(\frac{x}{\alpha}\right)^{\beta-1}
\exp\!\left[-\left(\frac{x}{\alpha}\right)^{\beta}\right] dx \\[6pt] &= \int_0^\infty 
\alpha^k u^{k/\beta}\,
\frac{\beta}{\alpha}\,
u^{(\beta-1)/\beta}\,
e^{-u}\,
\alpha \frac{1}{\beta}u^{1/\beta -1}\,du \\[6pt] &= \alpha^k \int_0^\infty 
u^{\frac{k+\beta}{\beta}-1} 
e^{-u}\,du = \alpha^k\,\,\Gamma\!\left(1+\frac{k}{\beta}\right)
\end{aligned}
\]
\end{frame}

\begin{frame}{Função Geradora de Momentos}
\phantomsection\label{funuxe7uxe3o-geradora-de-momentos}
\begin{tcolorbox}[enhanced jigsaw, toprule=.15mm, arc=.35mm, toptitle=1mm, leftrule=.75mm, opacityback=0, left=2mm, titlerule=0mm, colframe=quarto-callout-note-color-frame, title=\textcolor{quarto-callout-note-color}{\faInfo}\hspace{0.5em}{Definição 02: Função Geradora de Momentos}, coltitle=black, breakable, colback=white, colbacktitle=quarto-callout-note-color!10!white, bottomrule=.15mm, rightrule=.15mm, bottomtitle=1mm, opacitybacktitle=0.6]

Seja \(X\) uma variável aleatória qualquer. A \textbf{função geradora de
momentos} (FGM) de \(X\), denotada por \(M_X\), é definida por

\[
M_X(t)=E(e^{tX}),
\]

para valores de \(t\) em um intervalo contendo \(0\) onde a experança
exista.

\end{tcolorbox}

\textbf{Importante:} a função geradora de momentos é função de \(t\).
Para ela existir basta que exista \(\varepsilon > 0\) tal que
\(E(e^{tX})\) esteja bem definida para qualquer
\(t \in (-\varepsilon, \varepsilon)\).
\end{frame}

\begin{frame}{Função Geradora de Momentos}
\phantomsection\label{funuxe7uxe3o-geradora-de-momentos-1}
Note que a \textbf{definição de função geradora de momentos} é feita
independente do tipo de variável, mas a forma de encontrá-la depende se
a variável for \textbf{discreta} ou \textbf{contínua}, isto é,

\begin{itemize}
\tightlist
\item
  Se \(X\) é discreta,
\end{itemize}

\[M_X(t) = E(e^{tX}) = \sum_{\forall x \in S_X} \, e^{tx} p_X(x)\]

\begin{itemize}
\tightlist
\item
  Se \(X\) é contínua,
\end{itemize}

\[M_X(t) = E(e^{tX}) = \int_{-\infty}^{\infty} \, e^{tx} f_X(x) \, dx\]
\end{frame}

\begin{frame}{Função Geradora de Momentos}
\phantomsection\label{funuxe7uxe3o-geradora-de-momentos-2}
\begin{tcolorbox}[enhanced jigsaw, toprule=.15mm, arc=.35mm, toptitle=1mm, leftrule=.75mm, opacityback=0, left=2mm, titlerule=0mm, colframe=quarto-callout-note-color-frame, title=\textcolor{quarto-callout-note-color}{\faInfo}\hspace{0.5em}{Teorema 01}, coltitle=black, breakable, colback=white, colbacktitle=quarto-callout-note-color!10!white, bottomrule=.15mm, rightrule=.15mm, bottomtitle=1mm, opacitybacktitle=0.6]

Suponha que a função geradora de momentos de \(X\) exista para
\(|t| < \varepsilon\), \(\varepsilon > 0\). Então, \(E(X^k)\) existe
para \(k = 1, 2, \cdots\) e temos:

\[E(X^k) = \dfrac{d^k}{dt^k} M_X(t) \Bigg|_{t=0}\]

ou seja, o \(k\)-ésimo momento de \(X\) é igual à derivada de ordem
\(k\) de \(M_X(t)\) avaliada em \(t = 0\).

\end{tcolorbox}

\textbf{Demonstração:} Suponha que a função geradora de momentos de
\(X\) exista para todo \(t\) tal que \(|t|<\varepsilon\), com
\(\varepsilon>0\), isto é,

\[
M_X(t)=E(e^{tX})<\infty,\qquad |t|<\varepsilon
\]
\end{frame}

\begin{frame}{Função Geradora de Momentos}
\phantomsection\label{funuxe7uxe3o-geradora-de-momentos-3}
Pela série de Maclaurin da função exponencial, para qualquer número real
\(y\) temos \[
e^{y} = \sum_{n=0}^{\infty} \frac{y^n}{n!}
\]

Aplicando isso a \(y=tX\), obtemos, para cada \(t\) com
\(|t|<\varepsilon\), \[
e^{tX} = \sum_{n=0}^{\infty} \frac{(tX)^n}{n!} = 1 + tX +\frac{(tX)^2}{2!} + \frac{(tX)^3}{3!} + \cdots
\]

Temos que \(M_X(t)=E(e^{tX})\). Logo, admitindo ser válido permutar soma
infinita e esperança, temos
\end{frame}

\begin{frame}{Função Geradora de Momentos}
\phantomsection\label{funuxe7uxe3o-geradora-de-momentos-4}
\[
M_X(t)
= E(e^{tX})
= E\left(\sum_{n=0}^{\infty} \frac{(tX)^n}{n!}\right)
= \sum_{n=0}^{\infty} \frac{t^n}{n!} E(X^n),
\qquad |t|<\varepsilon
\]

Portanto, \(M_X(t)\) é dada, em uma vizinhança de \(0\), por uma série
de potência da forma \[
M_X(t) = \sum_{n=0}^{\infty} a_n t^n,
\qquad \text{com } a_n = \frac{E(X^n)}{n!}
\]
\end{frame}

\begin{frame}{Função Geradora de Momentos}
\phantomsection\label{funuxe7uxe3o-geradora-de-momentos-5}
Da teoria de séries de potência, sabemos que, se \[
M_X(t) = \sum_{n=0}^{\infty} a_n t^n,
\] então a \(k\)-ésima derivada é \[
M_X^{(k)}(t) = \sum_{n=k}^{\infty} a_n\, n(n-1)\cdots (n-k+1)\, t^{\,n-k}
\]
\end{frame}

\begin{frame}{Função Geradora de Momentos}
\phantomsection\label{funuxe7uxe3o-geradora-de-momentos-6}
Agora substituímos \(t=0\):

Observe:

\begin{itemize}
\tightlist
\item
  Se \(n > k\), aparece o fator \(t^{n-k} = 0^{n-k} = 0\);\\
\item
  Então \textbf{todos} os termos com \(n>k\) desaparecem;
\item
  Só o termo com \(n=k\) permanece.
\end{itemize}

O único termo sobrevivente é:

\[
a_k \, k(k-1)(k-2)\cdots 1 \, t^{\,0}
= a_k \, k!
\]
\end{frame}

\begin{frame}{Função Geradora de Momentos}
\phantomsection\label{funuxe7uxe3o-geradora-de-momentos-7}
Assim,

\[
M_X^{(k)}(0) = a_k\, k!
= \frac{E(X^k)}{k!}\,k!
= E(X^k)
\]

Logo, \[
E(X^k) = \left.\frac{d^k}{dt^k} M_X(t)\right|_{t=0},
\qquad k=1,2,\dots
\]

o que mostra que o \(k\)-ésimo momento de \(X\) é igual à derivada de
ordem \(k\) da função geradora de momentos avaliada em \(t=0\).
\end{frame}

\begin{frame}{Função Geradora de Momentos}
\phantomsection\label{funuxe7uxe3o-geradora-de-momentos-8}
\begin{tcolorbox}[enhanced jigsaw, toprule=.15mm, arc=.35mm, toptitle=1mm, leftrule=.75mm, opacityback=0, left=2mm, titlerule=0mm, colframe=quarto-callout-tip-color-frame, title=\textcolor{quarto-callout-tip-color}{\faLightbulb}\hspace{0.5em}{Dica Importante!}, coltitle=black, breakable, colback=white, colbacktitle=quarto-callout-tip-color!10!white, bottomrule=.15mm, rightrule=.15mm, bottomtitle=1mm, opacitybacktitle=0.6]

Para qualquer variável aleatória \(X\):

\[
M_X(0) = E(e^{0X}) = 1.
\]

Isso sempre deve ocorrer. Use esse fato para verificar se sua FGM está
correta.

\end{tcolorbox}
\end{frame}

\begin{frame}{Função Geradora de Momentos}
\phantomsection\label{funuxe7uxe3o-geradora-de-momentos-9}
\begin{tcolorbox}[enhanced jigsaw, toprule=.15mm, arc=.35mm, toptitle=1mm, leftrule=.75mm, opacityback=0, left=2mm, titlerule=0mm, colframe=quarto-callout-note-color-frame, title=\textcolor{quarto-callout-note-color}{\faInfo}\hspace{0.5em}{Exemplo 03: Distribuição de Bernoulli}, coltitle=black, breakable, colback=white, colbacktitle=quarto-callout-note-color!10!white, bottomrule=.15mm, rightrule=.15mm, bottomtitle=1mm, opacitybacktitle=0.6]

Seja \(X \sim Bernoulli(p)\). Encontre sua função geradora de momentos e
a partir dela, encontre \(E(X)\) e \(\operatorname{Var}(X)\).

\end{tcolorbox}

\pause

\textbf{Solução:} Por definição,

\[
M_X(t) = E(e^{tX}) = \sum_{x=0}^1 e^{tx} p^x(1-p)^{1-x}
\]

Como \(X\) só assume os valores \(0\) e \(1\):

\[
M_X(t)
= e^{t\cdot 0} p^0(1-p)^{1-0} + e^{t\cdot 1} p^1(1-p)^{1-1}
= (1-p) + e^{t}p
= 1 - p +  e^{t} p
\]
\end{frame}

\begin{frame}{Função Geradora de Momentos}
\phantomsection\label{funuxe7uxe3o-geradora-de-momentos-10}
Portanto,

\[
\boxed{M_X(t) = 1 - p + p e^{t}, \quad t \in \mathbb{R}}
\]

Veja que \(M_X(0) = 1 - p + p e^{0} = 1 - p + p \times 1 = 1\). Assim,

\begin{itemize}
\tightlist
\item
  Primeira derivada de \(M_X(t)\):
\end{itemize}

\[
M_X'(t) = \frac{d}{dt}\big(1 - p + p e^{t}\big)
= p e^{t}
\]

Avaliada em \(t = 0\), temos \(E(X) = M_X'(0) = p e^{0} = p\)
\end{frame}

\begin{frame}{Função Geradora de Momentos}
\phantomsection\label{funuxe7uxe3o-geradora-de-momentos-11}
\begin{itemize}
\tightlist
\item
  Segunda derivada de \(M_X(t)\):
\end{itemize}

\[
M_X''(t) = \frac{d}{dt}\big(M_X'(t)\big)
= \frac{d}{dt}(p e^{t}) = p e^{t}
\]

Avaliada em \(t = 0\), temos \(E(X) = M_X''(0) = p e^{0} = p\).

Assim,

\[
\operatorname{Var}(X)
= M_X''(0) - [M_X'(0)]^2
= p - p^2
= p(1-p)
\]
\end{frame}

\begin{frame}{Função Geradora de Momentos}
\phantomsection\label{funuxe7uxe3o-geradora-de-momentos-12}
\begin{tcolorbox}[enhanced jigsaw, toprule=.15mm, arc=.35mm, toptitle=1mm, leftrule=.75mm, opacityback=0, left=2mm, titlerule=0mm, colframe=quarto-callout-note-color-frame, title=\textcolor{quarto-callout-note-color}{\faInfo}\hspace{0.5em}{Exemplo 04: Distribuição Binomial}, coltitle=black, breakable, colback=white, colbacktitle=quarto-callout-note-color!10!white, bottomrule=.15mm, rightrule=.15mm, bottomtitle=1mm, opacitybacktitle=0.6]

Seja \(X \sim Binomial(n,p)\). Encontre sua função geradora de momentos
e a partir dela, encontre \(E(X)\) e \(\operatorname{Var}(X)\).

\end{tcolorbox}

\pause

\textbf{Solução:} Se \(X \sim Binomial(n,p)\) então sua f.p. é dada por

\[
P(X = k) = \binom{n}{k} p^k (1-p)^{n-k}, \qquad k=0,1,\dots,n
\]

Por definição,

\[
M_X(t) = E(e^{tX}) = \sum_{k=0}^{n} e^{tk} P(X = k)
= \sum_{k=0}^{n} e^{tk} \binom{n}{k} p^k (1-p)^{n-k}
\]
\end{frame}

\begin{frame}{Função Geradora de Momentos}
\phantomsection\label{funuxe7uxe3o-geradora-de-momentos-13}
Escrevendo \(e^{tk} = (e^{t})^k\):

\[
M_X(t)
= \sum_{k=0}^{n} \binom{n}{k} (p e^t)^k (1-p)^{n-k}
\]

Temos que o binômio de Newton é dado por:

\[
(a+b)^n = \sum_{k=0}^{n} \binom{n}{k} a^k b^{n-k}
\]
\end{frame}

\begin{frame}{Função Geradora de Momentos}
\phantomsection\label{funuxe7uxe3o-geradora-de-momentos-14}
Assim, tomando \(a = p e^{t}\) e \(b = 1-p\):

\[
M_X(t) = (1-p + p e^{t})^n
\]

Portanto,

\[
\boxed{M_X(t) = (1-p + p e^{t})^n,\quad t\in\mathbb{R}}
\]

Veja que,
\(M_X(0) = (1-p + p e^{0})^n = (1 - p + p \times 1)^n = 1^n = 1\).
\end{frame}

\begin{frame}{Função Geradora de Momentos}
\phantomsection\label{funuxe7uxe3o-geradora-de-momentos-15}
Sabemos que

\[
E(X) = M_X'(0)
\]

Derivada primeira em relação a \(t\),

\[
M_X'(t) = \frac{d}{dt} (1-p + p e^{t})^n
= n(1-p + p e^{t})^{n-1} \cdot p e^{t}
\]

Logo,

\[
M_X'(0)
= n(1-p + p e^{0})^{n-1} \cdot p e^{0}
= n(1-p + p)^{n-1} p
= n p
\]
\end{frame}

\begin{frame}{Função Geradora de Momentos}
\phantomsection\label{funuxe7uxe3o-geradora-de-momentos-16}
Vamos agora encontrar a segunda derivada. Temos

\[
M_X'(t) = n p e^{t} (1-p + p e^{t})^{n-1}
\]

Aplicando a regra do produto:

\[
\begin{aligned}
M_X''(t)
&= \frac{d}{dt}\Big[ n p e^{t} (1-p + p e^{t})^{n-1} \Big] \\
&= n p e^{t} (n-1)(1-p + p e^{t})^{n-2} \cdot p e^{t}
 \;+\; n p e^{t} (1-p + p e^{t})^{n-1}
\end{aligned}
\]
\end{frame}

\begin{frame}{Função Geradora de Momentos}
\phantomsection\label{funuxe7uxe3o-geradora-de-momentos-17}
Avaliando em \(t=0\):

\begin{itemize}
\tightlist
\item
  \(e^{0}=1\)\\
\item
  \(1-p + p e^{0} = 1-p + p = 1\)
\end{itemize}

Assim,

\[
\begin{aligned}
M_X''(0)
&= n p \cdot 1 \cdot (n-1) \cdot 1^{\,n-2} \cdot p \cdot 1
   \;+\; n p \cdot 1 \cdot 1^{\,n-1} \\[6pt]
&= n p (n-1)p + n p \\[6pt]
&= n p \big[(n-1)p + 1\big]
\end{aligned}
\]
\end{frame}

\begin{frame}{Função Geradora de Momentos}
\phantomsection\label{funuxe7uxe3o-geradora-de-momentos-18}
De forma que,

\[
\begin{aligned}
\operatorname{Var}(X)
&= M_X''(0) - [M_X'(0)]^2\\[6pt] 
&= n p \big[(n-1)p + 1\big] - (np)^2 \\[6pt]
&= n p \big[(n-1)p + 1 - n p\big] \\[6pt]
&= n p (1 - p)
\end{aligned}
\]
\end{frame}

\begin{frame}{Função Geradora de Momentos}
\phantomsection\label{funuxe7uxe3o-geradora-de-momentos-19}
\begin{tcolorbox}[enhanced jigsaw, toprule=.15mm, arc=.35mm, toptitle=1mm, leftrule=.75mm, opacityback=0, left=2mm, titlerule=0mm, colframe=quarto-callout-note-color-frame, title=\textcolor{quarto-callout-note-color}{\faInfo}\hspace{0.5em}{Exemplo 05: Distribuição Geométrica}, coltitle=black, breakable, colback=white, colbacktitle=quarto-callout-note-color!10!white, bottomrule=.15mm, rightrule=.15mm, bottomtitle=1mm, opacitybacktitle=0.6]

Seja \(X \sim Geo(p)\). Encontre sua função geradora de momentos e a
partir dela, encontre \(E(X)\) e \(\operatorname{Var}(X)\).

\end{tcolorbox}

\pause

\textbf{Solução:} Seja \(X \sim \text{Geo}(p)\), cuja função de
probabilidade é

\[
P(X = k) = p(1-p)^{k-1}, \qquad k = 1,2,3,\dots
\]

Por definição, temos que,

\[
M_X(t)
= E(e^{tX})
= \sum_{k=1}^{\infty} e^{tk} p(1-p)^{k-1} = p \sum_{k=1}^{\infty} (e^t)^k (1-p)^{k-1}
\]
\end{frame}

\begin{frame}{Função Geradora de Momentos}
\phantomsection\label{funuxe7uxe3o-geradora-de-momentos-20}
Reorganizando,

\[
M_X(t)
= p e^t \sum_{k=1}^{\infty} \big[(1-p)e^{t}\big]^{\,k-1}
\]

A soma é uma série geométrica com razão

\[
r=(1-p)e^{t}
\]

A série converge se e somente se:

\[
|(1-p)e^{t}| < 1
\]
\end{frame}

\begin{frame}{Função Geradora de Momentos}
\phantomsection\label{funuxe7uxe3o-geradora-de-momentos-21}
Usando a soma da série geométrica,

\[
\sum_{k=0}^{\infty} r^k = \frac{1}{1-r},
\]

temos:

\[
M_X(t) = \frac{p e^t}{1 - (1-p)e^{t}}
\]
\end{frame}

\begin{frame}{Função Geradora de Momentos}
\phantomsection\label{funuxe7uxe3o-geradora-de-momentos-22}
Portanto,

\[
\boxed{M_X(t) = \frac{p e^{t}}{1-(1-p)e^{t}}, \quad \text{para } t < \ln\Bigg(\frac{1}{1-p}\Bigg)}
\]

Note que,
\(M_X(0) = \dfrac{p e^{0}}{1-(1-p)e^{0}} = \dfrac{p}{1-(1-p)} =  \dfrac{p}{p} =1\)

\pause

Sabemos que:

\[
E(X) = M_X'(0)
\]
\end{frame}

\begin{frame}{Função Geradora de Momentos}
\phantomsection\label{funuxe7uxe3o-geradora-de-momentos-23}
Vamos então encontrar a derivada de primeira ordem de

\[
M_X(t) = \frac{p e^{t}}{1 - (1-p)e^{t}}
\]

Use regra do quociente, temos

\[
M_X'(t)
= \frac{p e^t \big[1 - (1-p)e^{t}\big] - p e^{t}\big[-(1-p)e^{t}\big]}{\big[1 - (1-p)e^{t}\big]^2}
\]

Simplificando o numerador:

\[
p e^{t}\left[1 - (1-p)e^{t} + (1-p)e^{t}\right]
= p e^{t}
\]
\end{frame}

\begin{frame}{Função Geradora de Momentos}
\phantomsection\label{funuxe7uxe3o-geradora-de-momentos-24}
Portanto,

\[
M_X'(t) = \frac{p e^{t}}{\left[1 - (1-p)e^{t}\right]^2}
\]

Avaliando em \(t=0\):

\begin{itemize}
\tightlist
\item
  \(e^0=1\)
\item
  \(1 - (1-p)e^{0} = 1-(1-p) = p\)
\end{itemize}

Logo,

\[E(X) = M_X'(0)
= \frac{p}{p^2}
= \frac{1}{p}\]
\end{frame}

\begin{frame}{Função Geradora de Momentos}
\phantomsection\label{funuxe7uxe3o-geradora-de-momentos-25}
Encontrando \(\operatorname{Var}(X)\), usamos,

\[
\operatorname{Var}(X) = M_X''(0) - \Big[M_X'(0)\Big]^2
\]

A derivada segunda é dada por

\[
M_X''(t)
= \frac{p e^{t}\big[1+(1-p)e^{t}\big]}{\left[1-(1-p)e^{t}\right]^3}
\]

de forma que

\[M_X''(0)
= \frac{p(1+(1-p))}{p^3}
= \frac{p(2-p)}{p^3}
= \frac{2-p}{p^2}\]
\end{frame}

\begin{frame}{Função Geradora de Momentos}
\phantomsection\label{funuxe7uxe3o-geradora-de-momentos-26}
e, portanto,

\[\operatorname{Var}(X)
= M_X''(0) - \Big[M_X'(0)\Big]^2
= \frac{2-p}{p^2} - \left(\frac{1}{p}\right)^2
= \frac{2-p-1}{p^2}
= \frac{1-p}{p^2}\]

\pause

Veja que neste exemplo, a função geradora de momentos de \(X ∼ Geo(p)\)
não está definida para todo \(t \in \mathbb{R}\), mas está bem definida
para \(t < \ln\Big(\frac{1}{1-p}\Big)\). E como
\(\ln\Big(\frac{1}{1-p}\Big) > 0\), a função geradora de momentos está
bem definida para \(t\) em uma vizinhança de zero.
\end{frame}

\begin{frame}{Função Geradora de Momentos}
\phantomsection\label{funuxe7uxe3o-geradora-de-momentos-27}
\begin{tcolorbox}[enhanced jigsaw, toprule=.15mm, arc=.35mm, toptitle=1mm, leftrule=.75mm, opacityback=0, left=2mm, titlerule=0mm, colframe=quarto-callout-note-color-frame, title=\textcolor{quarto-callout-note-color}{\faInfo}\hspace{0.5em}{Exemplo 06: Distribuição de Poisson}, coltitle=black, breakable, colback=white, colbacktitle=quarto-callout-note-color!10!white, bottomrule=.15mm, rightrule=.15mm, bottomtitle=1mm, opacitybacktitle=0.6]

Seja \(X \sim Poisson(\lambda)\). Encontre sua função geradora de
momentos e a partir dela, encontre \(E(X)\) e \(\operatorname{Var}(X)\).

\end{tcolorbox}

\pause

\textbf{Solução:} Seja \(X \sim Poisson(\lambda)\), cuja função de
probabilidade é

\[
P(X = k) = \frac{e^{-\lambda}\lambda^k}{k!}, \quad k = 0,1,2,\ldots
\]

Assim, a função geradora de momentos (fgm) de \(X\) é dada por,

\[
M_X(t) = \sum_{k=0}^\infty e^{tk}\,\frac{e^{-\lambda}\lambda^k}{k!}
       = e^{-\lambda} \sum_{k=0}^\infty \frac{(\lambda e^t)^k}{k!}
\]
\end{frame}

\begin{frame}{Função Geradora de Momentos}
\phantomsection\label{funuxe7uxe3o-geradora-de-momentos-28}
A série

\[
\sum_{k=0}^\infty \frac{(\lambda e^t)^k}{k!}
\]

é a expansão em série de Taylor de \(e^{\lambda e^t}\). Assim,

\[
M_X(t) = e^{-\lambda} e^{\lambda e^t}
       = \exp\{\lambda(e^t - 1)\}
\]
\end{frame}

\begin{frame}{Função Geradora de Momentos}
\phantomsection\label{funuxe7uxe3o-geradora-de-momentos-29}
Portanto, a fgm de \(X\) é \[
\boxed{M_X(t) = \exp\big(\lambda(e^t - 1)\big)}
\]

Note que,
\(M_X(0) = \exp\big(\lambda(e^0 - 1)\big) = \exp\big(0\big) =  1\)

\pause

Vamos calcular as derivadas de ordem primeira e ordem segunda para
calcular os respectivos momentos:

Pela regra da cadeia,

\[
M_X'(t) = \exp\big(\lambda(e^t - 1)\big)\cdot \lambda e^t
        = \lambda e^t\, M_X(t)
\]
\end{frame}

\begin{frame}{Função Geradora de Momentos}
\phantomsection\label{funuxe7uxe3o-geradora-de-momentos-30}
Logo,

\[
E(X) = M_X'(0) = \lambda e^0\, M_X(0) = \lambda \cdot 1 \cdot 1 = \lambda
\]

\pause

A segunda derivada de \(M_X(t)\) é dada por:

\[
\begin{aligned}
M_X''(t) 
&= \frac{d}{dt}\big[\lambda e^t\, M_X(t)\big] \\
&= \lambda e^t\, M_X(t) + \lambda e^t\, M_X'(t) \\
&= \lambda e^t\, M_X(t) + \lambda e^t\,[\lambda e^t\, M_X(t)] \\
&= \lambda e^t\, M_X(t)\,\big[1 + \lambda e^t\big]
\end{aligned}
\]
\end{frame}

\begin{frame}{Função Geradora de Momentos}
\phantomsection\label{funuxe7uxe3o-geradora-de-momentos-31}
Agora, avaliando em \(t=0\):

\[
\begin{aligned}
M_X''(0) = \lambda e^0\, M_X(0)\,\big[1 + \lambda e^0\big] = \lambda \cdot 1 \cdot 1 \cdot (1+\lambda) = \lambda(1+\lambda)
\end{aligned}
\]

Portanto,

\[
E(X^2) = M_X''(0) = \lambda(1+\lambda)
\]

Assim,

\[
\begin{aligned}
\text{Var}(X) 
&= M_X''(0) - \Big[M_X'(0)\Big]^2 \\
&= \lambda(1+\lambda) - \lambda^2 \\
&= \lambda
\end{aligned}
\]
\end{frame}

\begin{frame}{Função Geradora de Momentos}
\phantomsection\label{funuxe7uxe3o-geradora-de-momentos-32}
\textbf{Exemplo 07 - Distribuição Exponencial:} Seja
\(X \sim exp(\lambda)\). Encontre sua função geradora de momentos e a
partir dela, encontre \(E(X)\) e \(\operatorname{Var}(X)\).

\pause

\textbf{Solução:} Seja \(X \sim exp(\lambda)\), cuja função densidade é

\[
f(x) = \lambda e^{-\lambda x}, \quad x > 0
\]

Neste caso, a função geradora de momentos (fgm) de \(X\) é definida por
\[
M_X(t) = E(e^{tX}) = \int_0^\infty e^{tx}\,\lambda e^{-\lambda x}\,dx = \int_0^\infty \lambda e^{-(\lambda - t)x} \, dx
\]

válida para \(t < \lambda\).
\end{frame}

\begin{frame}{Função Geradora de Momentos}
\phantomsection\label{funuxe7uxe3o-geradora-de-momentos-33}
Assim, como

\[
\int_0^\infty \lambda e^{-(\lambda - t)x} \,dx =  \lambda \frac{e^{-(\lambda - t)x}}{-(\lambda - t)} \Bigg|_0^{\infty} = \frac{\lambda}{\lambda - t}, \,\,\,\, \lambda - t > 0
\]

Logo,

\[
\boxed{M_X(t) = \frac{\lambda}{\lambda - t}, \qquad t < \lambda}
\]

Note que, \(M_X(0) = \frac{\lambda}{\lambda - 0} =  1\)
\end{frame}

\begin{frame}{Função Geradora de Momentos}
\phantomsection\label{funuxe7uxe3o-geradora-de-momentos-34}
Derivando,

\[
  M_X'(t) = \frac{\lambda}{(\lambda - t)^2}
\]

Portanto,

\[
E(X) = M_X'(0) = \frac{\lambda}{\lambda^2} = \frac{1}{\lambda}
\]
\end{frame}

\begin{frame}{Função Geradora de Momentos}
\phantomsection\label{funuxe7uxe3o-geradora-de-momentos-35}
A segunda derivada é dada por,

\[
M_X''(t) = \frac{2\lambda}{(\lambda - t)^3}
\]

Logo,

\[
E(X^2) = M_X''(0) = \frac{2\lambda}{\lambda^3} = \frac{2}{\lambda^2}
\]

De forma que,

\[
\text{Var}(X)
= E(X^2) - \Big[E(X) \Big]^2
= \frac{2}{\lambda^2} - \frac{1}{\lambda^2}
= \frac{1}{\lambda^2}
\]
\end{frame}

\begin{frame}{Função Geradora de Momentos}
\phantomsection\label{funuxe7uxe3o-geradora-de-momentos-36}
\textbf{Exemplo 08 - Distribuição Gamma:} Seja
\(X \sim Gamma(\alpha, \lambda)\). Encontre sua função geradora de
momentos.

\pause

\textbf{Solução:} Seja \(X \sim Gamma(\alpha, \lambda)\), cuja função
densidade é

\[
  f(x) = \frac{\lambda^\alpha}{\Gamma(\alpha)}x^{\alpha-1}e^{-\lambda x},
\qquad x>0
\]

A função geradora de momentos é

\[
  M_X(t)
= \int_0^\infty e^{tx}
\frac{\lambda^\alpha}{\Gamma(\alpha)}x^{\alpha-1}e^{-\lambda x}\,dx
= \frac{\lambda^\alpha}{\Gamma(\alpha)}
\int_0^\infty x^{\alpha-1}e^{-(\lambda - t)x}\,dx
\]
\end{frame}

\begin{frame}{Função Geradora de Momentos}
\phantomsection\label{funuxe7uxe3o-geradora-de-momentos-37}
Usamos agora o resultado da função Gamma:

\[
\int_0^\infty x^{\alpha-1}e^{-bx}\,dx
= \frac{\Gamma(\alpha)}{b^\alpha}, \qquad b>0
\]

Aqui, \(b = \lambda - t\) (precisamos de \(\lambda - t > 0\), isto é,
\(t<\lambda\)). Logo,

\[
\int_0^\infty x^{\alpha-1}e^{-(\lambda - t)x}\,dx
= \frac{\Gamma(\alpha)}{(\lambda - t)^\alpha}
\]

Portanto,

\[
M_X(t)
= \frac{\lambda^\alpha}{\Gamma(\alpha)}
\cdot
\frac{\Gamma(\alpha)}{(\lambda - t)^\alpha}
= \left(\frac{\lambda}{\lambda - t}\right)^\alpha,
\qquad t < \lambda
\]
\end{frame}

\begin{frame}{Função Geradora de Momentos}
\phantomsection\label{funuxe7uxe3o-geradora-de-momentos-38}
\textbf{Exemplo 09 - Distribuição Normal Padrão:} Seja
\(X \sim N(0, 1)\). Encontre sua função geradora de momentos.

\pause

\textbf{Solução:} Seja \(X \sim N(0, 1)\), cuja função densidade é

\[
f(x) = \frac{1}{\sqrt{2\pi}} e^{-x^2/2}, \qquad x \in \mathbb{R}
\]

Queremos encontrar a função geradora de momentos (fgm) dada por

\[
M_X(t)
= \int_{-\infty}^{\infty} e^{tx}\,\frac{1}{\sqrt{2\pi}} e^{-x^2/2}\,dx
= \frac{1}{\sqrt{2\pi}}\int_{-\infty}^{\infty}
\exp\Big(tx - \frac{x^2}{2}\Big)\,dx
\]
\end{frame}

\begin{frame}{Função Geradora de Momentos}
\phantomsection\label{funuxe7uxe3o-geradora-de-momentos-39}
Vamos usar a técnica de completar quadrados no expoente,

\[
\begin{aligned}
tx - \frac{x^2}{2}
&= -\frac{1}{2}\big(x^2 - 2tx\big) = -\frac{1}{2}\big(x^2 - 2tx + t^2 - t^2\big) \\
&= -\frac{1}{2}\big[(x - t)^2 - t^2\big] = -\frac{(x - t)^2}{2} + \frac{t^2}{2}
\end{aligned}
\]

Assim,

\[\small
M_X(t)
= \frac{1}{\sqrt{2\pi}}\int_{-\infty}^{\infty}
\exp\!\left(-\frac{(x - t)^2}{2} + \frac{t^2}{2}\right)\,dx
= \exp\!\left(\frac{t^2}{2}\right)
\frac{1}{\sqrt{2\pi}}\int_{-\infty}^{\infty}
\exp\!\left(-\frac{(x - t)^2}{2}\right)\,dx
\]
\end{frame}

\begin{frame}{Função Geradora de Momentos}
\phantomsection\label{funuxe7uxe3o-geradora-de-momentos-40}
A integral

\[
\frac{1}{\sqrt{2\pi}}\int_{-\infty}^{\infty}
\exp\!\left(-\frac{(x - t)^2}{2}\right)\,dx
\] é a área total sob a curva de uma normal \(N(t,1)\), e portanto é
igual a \(1\).

Logo, \[
\boxed{M_X(t) = \exp\!\left(\frac{t^2}{2}\right), \qquad t \in \mathbb{R}}
\]
\end{frame}

\begin{frame}{Função Geradora de Momentos}
\phantomsection\label{funuxe7uxe3o-geradora-de-momentos-41}
\textbf{Proposição 01 - Transformações Lineares:} Seja \(X\) uma
variável aleatória com função geradora de momentos \(M_X\). Seja
\(Y = aX + b\). Então, a função geradora de momentos de \(Y\) é dada por

\[M_Y(t) = e^{bt} M_X(at)\]

\pause

\textbf{Demonstração:}

\[M_Y(t) = E(e^{tY}) = E(e^{t(aX+b)}) = E\Big(e^{atX} e^{bt} \Big) = e^{bt} E\Big(e^{atX}\Big) =  e^{bt} M_X(at)\]
\end{frame}

\begin{frame}{Função Geradora de Momentos}
\phantomsection\label{funuxe7uxe3o-geradora-de-momentos-42}
\textbf{Exemplo 10 - Distribuição Normal:} Seja
\(X \sim N(\mu, \sigma^2)\). Encontre sua função geradora de momentos.

\pause

\textbf{Solução:} Se \(X \sim N(\mu, \sigma^2)\), então
\(X = \sigma Z + \mu\), em que \(Z\sim N(0,1)\), com função geradora de
momentos dada por \(M_Z(t) = \exp\!\left(\frac{t^2}{2}\right)\). Assim,

\[M_X(t) = e^{\mu t} M_Z(\sigma t) = e^{\mu t} \exp\!\left(\frac{(\sigma t)^2}{2}\right) = e^{\mu t} \exp\!\left(\frac{\sigma^2 t^2}{2}\right)\]
\end{frame}

\begin{frame}{Função Geradora de Momentos}
\phantomsection\label{funuxe7uxe3o-geradora-de-momentos-43}
Logo,

Se \(X \sim N(\mu, \sigma^2)\), então

\[\boxed{M_X(t)= \exp \Bigg(\mu t + \frac{\sigma^2 t^2}{2} \Bigg)}\]
\end{frame}

\begin{frame}{Função Geradora de Momentos}
\phantomsection\label{funuxe7uxe3o-geradora-de-momentos-44}
\textbf{Teorema 02:} Se duas variáveis aleatórias têm funções geradoras
de momentos que existem, e são iguais, então elas têm a mesma função de
distribuição.

\pause

A demonstração desse Teorema será omitida, pois ela usa conceitos não
estudados neste curso. Veja que o Teorema 02 nos mostra que se duas
variáveis aleatórias tem mesma função geradora de momentos então estas
variáveis aleatórias são identicamente distribuídas.

Isso significa que podemos identificar a distribuição de uma variável
aleatória a partir da sua função geradora de momentos, assim como
identificamos a distribuição da variável aleatória a partir da sua
função de distribuição, função densidade ou função de probabilidade.
\end{frame}

\begin{frame}{Função Geradora de Momentos}
\phantomsection\label{funuxe7uxe3o-geradora-de-momentos-45}
\textbf{Exemplo 11:} Seja \(X \sim Gamma(\alpha, \lambda)\). Considere
também \(Y = cX,\,\, c\in \mathbb{R}\). Mostre, a partir da função
geradora de momentos, que \(Y \sim Gamma(\alpha, \lambda/c)\).

\pause

\textbf{Solução:} Por definição,

\[
M_Y(t) = E(e^{tY}) = E(e^{t(cX)}) = E(e^{(ct)X})
\]

Logo,

\[
M_Y(t) = M_X(ct) = \left(\frac{\lambda}{\lambda - ct}\right)^{\alpha},
\qquad ct < \lambda \;\; \Rightarrow \;\; t < \frac{\lambda}{c}
\]
\end{frame}

\begin{frame}{Função Geradora de Momentos}
\phantomsection\label{funuxe7uxe3o-geradora-de-momentos-46}
ou seja,

\[
M_Y(t)
= \left(\frac{\lambda}{\lambda - ct}\right)^{\alpha}
= \left(\frac{\lambda/c}{(\lambda/c) - t}\right)^{\alpha},
\qquad t < \frac{\lambda}{c}
\]

Veja que trata-se da função geradora de momentos de uma variável
aleatória com distribuição \(Gamma(\alpha,\lambda/c)\). Logo,
\(Y \sim Gamma(\alpha,\lambda/c)\).
\end{frame}

\begin{frame}{Função Geradora de Momentos}
\phantomsection\label{funuxe7uxe3o-geradora-de-momentos-47}
\textbf{Teorema 03 - Soma de variáveis independentes:} Sejam
\(X_1, X_2, \cdots, X_n\) variáveis aleatórias independentes e funções
geradoras de momentos, respectivamente, iguais a
\(M_{X_j}(t), \,\, j = 1, 2, \cdots, n\) para \(t\) em alguma vizinhança
de zero. Se \(Y = X_1 + X_2 + \cdots + X_n\), então a função geradora de
momentos de \(Y\) existe e é dada por:

\[M_Y(t) =  \prod_{j = 1}^n M_{X_j}(t)\]
\end{frame}

\begin{frame}{Função Geradora de Momentos}
\phantomsection\label{funuxe7uxe3o-geradora-de-momentos-48}
\textbf{Demonstração:} Pela definição, temos

\[
\begin{aligned}
M_Y(t) &= E(e^{t(X_1 + X_2 + \cdots + X_n)}) \\ &= E(e^{tX_1}e^{tX_2} \cdots e^{tX_n}) \\ &= E(e^{tX_1})E(e^{tX_2})\cdots E(e^{tX_n})
\end{aligned}
\] e daí segue o resultado desejado.
\end{frame}

\begin{frame}{Função Geradora de Momentos}
\phantomsection\label{funuxe7uxe3o-geradora-de-momentos-49}
\textbf{Exemplo 12:} Sejam \(X_1, X_2, \cdots, X_n\) variáveis
aleatórias independentes com distribuição Bernoulli de parâmetro \(p\).
Se \(Y = X_1 + X_2 + \cdots + X_n\), mostre, a partir da função geradora
de momentos, que \(Y \sim Binomial(n,p)\).

\pause

\textbf{Solução:} Temos que se \(X_1, X_2, \cdots, X_n\) são variáveis
aleatórias independentes com distribuição Bernoulli de parâmetro \(p\),
então

\[
M_{X_j}(t) = 1 - p + p e^{t}, \quad j = 1,2, \cdots n
\]
\end{frame}

\begin{frame}{Função Geradora de Momentos}
\phantomsection\label{funuxe7uxe3o-geradora-de-momentos-50}
Então, pelo \textbf{Teorema 03}, temos

\[
M_Y(t) = \prod_{j = 1}^n M_{X_j}(t) = \big(1 - p + p e^{t}\big)^n
\]

que corresponde à função geradora de momentos de uma variável aleatória
Binomial com parâmetros \(n\) e \(p\). Logo, temos que
\(X_1 + X_2 + \cdots + X_n \sim Binomial(n,p)\).
\end{frame}

\begin{frame}{Função Característica}
\phantomsection\label{funuxe7uxe3o-caracteruxedstica}
A \textbf{função característica} é uma das ferramentas mais importantes
da Teoria das Probabilidades.

\pause

Ela desempenha um papel similar ao da função geradora de momentos (fgm),
mas com vantagens significativas: \textbf{sempre existe}, determina
unicamente a distribuição e facilita o estudo de somas de variáveis
aleatórias.

\pause

\textbf{Definição 03 - Variáveis Aleatórias Complexas:} Dizemos que uma
variável aleatória \(X\) é complexa, se pode ser escrita como

\[X = X^a + i \,X^b\]

em que \(i = \sqrt{-1}\) e \(X^a\) e \(X^b\) são variáveis aleatórias
reais.
\end{frame}

\begin{frame}{Função Característica}
\phantomsection\label{funuxe7uxe3o-caracteruxedstica-1}
Para o valor esperado de \(X\), no caso complexo, exigimos que as
esperanças das duas partes sejam finitas. Assim, temos:

\[E(X) = E(X^a) + i \, E(X^b)\]

em que \(E(X^a)\) e \(E(X^b)\) são finitas.

\pause

Para efeitos práticos, podemos trabalhar funções complexas como se
estivéssemos com funções reais.

\pause

As propriedades usuais de números complexos serão usadas. Sendo
\(z = a + i\, b\), denotaremos por \(|z|\) seu módulo e por \(\bar{z}\)
seu conjugado.
\end{frame}

\begin{frame}{Função Característica}
\phantomsection\label{funuxe7uxe3o-caracteruxedstica-2}
\textbf{Definição 04 - Função Característica:} A função característica
de uma variável aleatória \(X\) é definida por

\[
\varphi_X(t) = E\!\left(e^{itX}\right) = E\Big[\cos(tX)\Big] + i\, E\Big[\, \text{sen}(tX)\Big],
\qquad t \in \mathbb{R},
\] onde \(i = \sqrt{-1}.\)
\end{frame}

\begin{frame}{Função Característica}
\phantomsection\label{funuxe7uxe3o-caracteruxedstica-3}
\begin{block}{Observações importantes}
\phantomsection\label{observauxe7uxf5es-importantes}
\begin{enumerate}
\item
  \textbf{Sempre existe}\\
  Diferentemente da fgm, que pode divergir,\\
  \[
  |e^{itX}| = 1 \quad \Rightarrow \quad |E(e^{itX})| \le 1.
  \] Portanto, \(\varphi_X(t)\) está sempre bem definida.
\item
  \textbf{Determina unicamente a distribuição}\\
  Se \(\varphi_X(t) = \varphi_Y(t)\) para todo \(t\), então\\
  \[
  X \stackrel{d}{=} Y.
  \]
\end{enumerate}
\end{block}
\end{frame}

\begin{frame}{Função Característica}
\phantomsection\label{funuxe7uxe3o-caracteruxedstica-4}
\begin{block}{Observações importantes}
\phantomsection\label{observauxe7uxf5es-importantes-1}
\begin{enumerate}
\setcounter{enumi}{2}
\item
  \textbf{É sempre contínua}\\
  Aliás, é uniformemente contínua em toda a reta real.
\item
  \textbf{Normalização}\\
  Como \(E(e^{it\cdot 0}) = 1\), temos \[
  \varphi_X(0) = 1
  \]
\end{enumerate}
\end{block}
\end{frame}

\begin{frame}{Função Característica}
\phantomsection\label{funuxe7uxe3o-caracteruxedstica-5}
\begin{block}{Relação com Momentos}
\phantomsection\label{relauxe7uxe3o-com-momentos}
Se \(X\) possui momentos até ordem \(k\), então:

\[
\varphi_X^{(n)}(0) = i^n E(X^n),
\qquad n = 1,2,\dots,k.
\]

Em particular:

\begin{itemize}
\tightlist
\item
  \(E(X) = -i\,\varphi_X'(0)\)\\
\item
  \(\text{Var}(X) = -\varphi_X''(0) - \left[\varphi_X'(0)\right]^2\)
\end{itemize}
\end{block}
\end{frame}

\begin{frame}{Função Característica}
\phantomsection\label{funuxe7uxe3o-caracteruxedstica-6}
\begin{block}{Propriedade fundamental: soma de variáveis independentes}
\phantomsection\label{propriedade-fundamental-soma-de-variuxe1veis-independentes}
Se \(X\) e \(Y\) são independentes:

\[
\varphi_{X+Y}(t)
= E(e^{it(X+Y)})
= E(e^{itX})\,E(e^{itY})
= \varphi_X(t)\,\varphi_Y(t).
\]

Esse resultado é crucial, por exemplo, para demonstrar o \textbf{Teorema
Central do Limite}.
\end{block}
\end{frame}

\begin{frame}{Função Característica}
\phantomsection\label{funuxe7uxe3o-caracteruxedstica-7}
\textbf{Exemplo 13:} Considere uma variável aleatória \(X\) com
distribuição Uniforme Contínua no intervalo \([a,b]\). Encontre sua
função característica.

\pause

\textbf{Solução:} Considere \(X \sim \text{U}(a,b)\), com \(a<b\). Temos
que, sua função densidade é dada por

\[
f(x) = \frac{1}{b-a}, \qquad a \le x \le b
\]

Assim, a função característica de \(X\) é

\[
\varphi_X(t) = E(e^{itX})
= \int_{-\infty}^{\infty} e^{itx} f(x)\,dx
\]
\end{frame}

\begin{frame}{Função Característica}
\phantomsection\label{funuxe7uxe3o-caracteruxedstica-8}
Como \(f(x)\) é não nula apenas em \([a,b]\):

\[
\varphi_X(t)
= \int_a^b e^{itx}\,\frac{1}{b-a}\,dx
= \frac{1}{b-a} \int_a^b e^{itx}\,dx.
\]

mudança de variável: \(u = itx \Rightarrow du = it \, dx\) ou
\(dx = \frac{1}{it} du\). Quando \(x = a \Rightarrow u = ita\), quando
\(x = b \Rightarrow u = itb\). Assim,

\[
\varphi_X(t)
= \frac{1}{b-a}
\left[\frac{1}{it}e^{u}\right]_{ita}^{itb}
= \frac{e^{itb} - e^{ita}}{it \big(b-a\big)},
\qquad t \ne 0
\]
\end{frame}

\begin{frame}{Função Característica}
\phantomsection\label{funuxe7uxe3o-caracteruxedstica-9}
\textbf{Exemplo 14:} Seja \(X \sim Gamma(\alpha, \lambda)\). Encontre
sua função característica.

\pause

\textbf{Solução:} Se \(X \sim Gamma(\alpha, \lambda)\). Temos que sua
função densidade é

\[
f(x) = \frac{\lambda^\alpha}{\Gamma(\alpha)}x^{\alpha-1}e^{-\lambda x},
\qquad x>0
\]

Assim, a função característica de \(X\) é

\[
\varphi_X(t) = E(e^{itX})
= \int_0^\infty e^{itx} f(x)\,dx
\]
\end{frame}

\begin{frame}{Função Característica}
\phantomsection\label{funuxe7uxe3o-caracteruxedstica-10}
Substituindo a densidade:

\[
\varphi_X(t)
= \int_0^\infty e^{itx}
\frac{\lambda^\alpha}{\Gamma(\alpha)}x^{\alpha-1}e^{-\lambda x}\,dx
= \frac{\lambda^\alpha}{\Gamma(\alpha)}
\int_0^\infty x^{\alpha-1} e^{-(\lambda - it)x}\,dx
\]

Sabemos que,

\[
\int_0^\infty x^{\alpha-1}e^{-bx}\,dx = \frac{\Gamma(\alpha)}{b^\alpha}
\]

Assim,

\[
\varphi_X(t)
= \frac{\lambda^\alpha}{\Gamma(\alpha)}
\cdot
\frac{\Gamma(\alpha)}{(\lambda - it)^\alpha}
= \left(\frac{\lambda}{\lambda - it}\right)^\alpha
\]
\end{frame}

\begin{frame}{Função Característica}
\phantomsection\label{funuxe7uxe3o-caracteruxedstica-11}
\begin{block}{Observações importantes}
\phantomsection\label{observauxe7uxf5es-importantes-2}
\begin{itemize}
\item
  A função característica \textbf{generaliza} a fgm para o plano
  complexo.
\item
  Sempre que a fgm existir, vale: \[
  \varphi_X(t) = M_X(it)
  \]
\item
  Ambas determinam a distribuição, mas a função característica é
  \textbf{mais poderosa}, pois \textbf{sempre existe} e se comporta
  melhor em operações como soma de variáveis independentes.
\end{itemize}
\end{block}
\end{frame}

\begin{frame}{Função Característica}
\phantomsection\label{funuxe7uxe3o-caracteruxedstica-12}
\textbf{Exemplo 15:} Seja \(X \sim Binomial(n, p)\). Encontre sua função
característica e, através dela, encontre \(E(X)\) e \(\text{Var}(X)\).

\pause

\textbf{Solução:} Se \(X \sim Binomial(n, p)\), sua função geradora de
momentos é dada por

\[
M_X(t) = (1-p + p e^{t})^n
\]

Portanto, segue que

\[
\varphi_X(t) = M_X(it) = (1-p + p e^{it})^n
\]
\end{frame}

\begin{frame}{Função Característica}
\phantomsection\label{funuxe7uxe3o-caracteruxedstica-13}
Em geral, se a função característica é \(\varphi_X(t)=E(e^{itX})\),
então

\[
\varphi_X'(0) = i\,E(X) 
\quad \Rightarrow \quad
E(X) = \frac{\varphi_X'(0)}{i}
\]

Note que,

\[
\varphi_X(t) = g(t)^n,
\quad\text{onde}\quad
g(t) = 1-p + pe^{it}
\]

e

\[
\varphi_X'(t) = n g(t)^{n-1} g'(t)
= n (1-p+pe^{it})^{n-1} \, p i e^{it}
\]
\end{frame}

\begin{frame}{Função Característica}
\phantomsection\label{funuxe7uxe3o-caracteruxedstica-14}
Avaliando em \(t=0\):

\[
\varphi_X'(0) = n (1-p+pe^{i 0})^{n-1} \, p i e^{i\, 0} = n (1-p+pe^{0})^{n-1} \, p i e^{0} = n (1-p+p)^{n-1} \, p i = n p i
\]

Portanto,

\[
E(X) = \frac{\varphi_X'(0)}{i} = \frac{i np}{i} = np
\]
\end{frame}

\begin{frame}{Função Característica}
\phantomsection\label{funuxe7uxe3o-caracteruxedstica-15}
Já temos \[
\varphi_X'(t) = n p i e^{it}\, g(t)^{n-1},
\quad g(t)=1-p+pe^{it}
\]

Derivando novamente (regra do produto):

\[
\begin{aligned}
\varphi_X''(t)
&= n p i \Big[ i e^{it} g(t)^{n-1}
      + e^{it}(n-1) g(t)^{n-2} g'(t) \Big] \\
&= n p i \Big[ i e^{it} g(t)^{n-1}
      + e^{it}(n-1) g(t)^{n-2} (p i e^{it}) \Big] \\
&= n p \Big[ i^2 e^{it} g(t)^{n-1}
      + (n-1)p i^2 e^{2it} g(t)^{n-2} \Big] \\
&= -n p \Big[ e^{it} g(t)^{n-1}
      + (n-1)p e^{2it} g(t)^{n-2} \Big]
\end{aligned}
\]
\end{frame}

\begin{frame}{Função Característica}
\phantomsection\label{funuxe7uxe3o-caracteruxedstica-16}
Em \(t=0\) (\(g(0)=1\), \(e^{0}=1\)):

\[
\varphi_X''(0)
= -n p \big[ 1\cdot 1^{n-1} + (n-1)p\cdot 1\cdot 1^{n-2} \big]
= -n p \big[ 1 + (n-1)p \big]
\]

Então,

\[
E(X^2) = -\varphi_X''(0)
= n p\big[1 + (n-1)p\big]
= np + n(n-1)p^2
\]
\end{frame}

\begin{frame}{Função Característica}
\phantomsection\label{funuxe7uxe3o-caracteruxedstica-17}
Por fim, como
\(\text{Var}(X) = -\varphi_X''(0) - \left[\varphi_X'(0)\right]^2\)

\[
\begin{aligned}
\text{Var}(X)
&= -\varphi_X''(0) - \left[\varphi_X'(0)\right]^2 \\
&= \big[np + n(n-1)p^2\big] - (np)^2 \\
&= np + n(n-1)p^2 - n^2p^2 \\
&= np + \big(n^2 - n - n^2\big)p^2 \\
&= np - np^2 \\
&= np(1-p)
\end{aligned}
\]
\end{frame}

\begin{frame}{Função Característica}
\phantomsection\label{funuxe7uxe3o-caracteruxedstica-18}
\textbf{Exemplo 16:} Seja \(X \sim N(\mu, \sigma^2)\). Encontre sua
função característica e, através dela, encontre \(E(X)\) e
\(\text{Var}(X)\).

\pause

\textbf{Solução:} Se \(X \sim N(\mu, \sigma^2)\), sua função geradora de
momentos é dada por

\[M_X(t)= \exp \Bigg(\mu t + \frac{\sigma^2 t^2}{2} \Bigg)\]

Portanto, segue que

\[
\varphi_X(t) = M_X(it) = \exp \Bigg(it\mu - \frac{\sigma^2 t^2}{2} \Bigg)
\]
\end{frame}

\begin{frame}{Função Característica}
\phantomsection\label{funuxe7uxe3o-caracteruxedstica-19}
Derivando:

\[
\varphi_X'(t)
= \left(i\mu - \sigma^2 t\right)\,
\exp\!\left(it\mu - \frac{\sigma^2 t^2}{2}\right)
= \left(i\mu - \sigma^2 t\right)\varphi_X(t)
\]

Em \(t=0\):

\[
\varphi_X(0) = 1, \qquad
\varphi_X'(0) = i\mu
\]

Logo,

\[
E(X) = \frac{\varphi_X'(0)}{i} = \frac{i\mu}{i} = \mu
\]
\end{frame}

\begin{frame}{Função Característica}
\phantomsection\label{funuxe7uxe3o-caracteruxedstica-20}
Derivamos novamente:

\[
\begin{aligned}
\varphi_X''(t)
&= \frac{d}{dt}\big[(i\mu - \sigma^2 t)\varphi_X(t)\big] \\
&= -\sigma^2 \varphi_X(t)
   + (i\mu - \sigma^2 t)\varphi_X'(t) \\
&= -\sigma^2 \varphi_X(t)
   + (i\mu - \sigma^2 t)^2 \varphi_X(t)
\end{aligned}
\]

Em \(t=0\): \[
\varphi_X(0) = 1,\qquad
\varphi_X''(0) = -\sigma^2 + (i\mu)^2
              = -\sigma^2 - \mu^2
\]

Portanto, \[
E(X^2) = -\varphi_X''(0)
       = \sigma^2 + \mu^2
\]
\end{frame}

\begin{frame}{Função Característica}
\phantomsection\label{funuxe7uxe3o-caracteruxedstica-21}
De forma que,

\[
\operatorname{Var}(X)
= E(X^2) - [E(X)]^2
= (\sigma^2 + \mu^2) - \mu^2
= \sigma^2
\]
\end{frame}




\end{document}
